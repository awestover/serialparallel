\documentclass{article}[11pt]
\usepackage[subtle]{savetrees}
\usepackage[left=1in, right=1in, top=1in, bottom=1in]{geometry}

\usepackage{amsthm}
\usepackage{amssymb}
\usepackage{amsmath}
\usepackage{mathtools}

\usepackage{fancyhdr}
\pagestyle{fancy}
\lhead{Alek Westover}
\rhead{}
\usepackage{hyperref}
\usepackage{xspace}

\usepackage{algorithm}
\usepackage{algpseudocode} % adding [noend] deletes the end while and stuff


\newcommand{\defn}[1]{{\textit{\textbf{\boldmath #1}}}\xspace}
\renewcommand{\paragraph}[1]{\vspace{0.09in}\noindent{\bf \boldmath #1.}} 

\DeclareMathOperator{\E}{\mathbb{E}}
\DeclareMathOperator{\Var}{\text{Var}}

\DeclareMathOperator{\img}{Img}
\DeclareMathOperator{\polylog}{\text{polylog}}
\DeclareMathOperator{\st}{\text{ such that }}
\newcommand{\norm}[1]{\left\lVert#1\right\rVert}
\newcommand{\interior}[1]{ {\kern0pt#1}^{\mathrm{o}} }
\newcommand{\mb}{\mathbf}
\newcommand{\x}{\mathbf{x}}
\newcommand{\y}{\mathbf{y}}
\newcommand{\z}{\mathbf{z}}
\renewcommand{\d}{\mathrm{d}} %straight d for integrals
\renewcommand{\Re}{\mathrm{Re}}
\renewcommand{\Im}{\mathrm{Im}}

\newcommand{\setof}[2]{\left\{ #1\; : \;#2 \right\}}
\newcommand{\set}[1]{\left\{ #1\right\}}

\newcommand{\R}{\mathbb{R}}
\newcommand{\C}{\mathbb{C}}
\newcommand{\Z}{\mathbb{Z}}
\newcommand{\contr}{\[ \Rightarrow\!\Leftarrow \]}
\newcommand{\defeq}{\vcentcolon=}
\newcommand{\eqdef}{=\vcentcolon}

\newcommand{\floor}[1]{\left\lfloor #1 \right\rfloor}
\newcommand{\ceil}[1]{\left\lceil #1 \right\rceil}
\newcommand{\paren}[1]{\left( #1 \right)}

\usepackage[capitalise,nameinlink,noabbrev]{cleveref}
\crefname{equation}{}{} % cref{eq:blah} only does (1) instead of Equation (1)
\crefname{enumi}{Step}{} % cref{eq:blah} only does (1) instead of Item(1)

\newtheorem{fact}{Fact}
\newtheorem{definition}{Definition}
\newtheorem{remark}{Remark}
\newtheorem{claim}{Claim}
\newtheorem{proposition}{Proposition}
\newtheorem{lemma}{Lemma}
\newtheorem{corollary}{Corollary}
\newtheorem{theorem}{Theorem}
\usepackage{mdframed}
\newmdtheoremenv{q}{Question}[section]
\newenvironment{ans}
  { \emph{Solution.}}
  {$ $\\\noindent\rule{\textwidth}{1pt}}


\begin{document}
\begin{center}
\begin{Large}
	Alek Westover \\
	\vspace{2mm}
	Math 
\end{Large}
\end{center}
\thispagestyle{empty}

\section{chill}
\begin{algorithm}
  \caption{Alg chill}
  \begin{algorithmic}[1]
    \Procedure{chill}{}
    \While{True}
      \If{all processors are idle}
        \If{If there are more than $p/k$ queued tasks}
          \State schedule all (or $p$) in serial
        \Else
          \State Schedule one in parallel
        \EndIf
      \EndIf
    \EndWhile
    \EndProcedure
  \end{algorithmic}
\end{algorithm}

You can probably do a similar thing for the non-symmetric case.
The main idea is locally be greedy, but also lazy about
scheduling.
So in the non-symmetric case you'd probably be like ``if all
processors are idle, schedule in a good way."


\section{Alg X}

% number of tasks queued $\bmod p$ is interesting.
% it's pretty clear that in the symmetric case if you accumulate
% $p$ queued tasks you should just run them all in serial, hence
% you should just forget about them
\begin{algorithm}
  \caption{Alg X}
  \label{alg:nonharmfultermination}
  \begin{algorithmic}[1]
    \State This procedure is continuously running.
    \Procedure{X}{}
    \If{there are more than $p/k$ queued tasks $\bmod p$}
      \For{each task running in parallel}
        \If{it has more than $1$ total work left \textbf{and} the number of queued tasks $\bmod p$ is at most $p-1$}
          \State kill this task it (i.e. put it on the queue)
        \EndIf
      \EndFor
      \State Schedule as many queued tasks as possible to
      processros that have less than $1$ work assigned to them.
      (scheduling to minimize backlog, i.e. scheduling in
      ascending order of backlog)
    \EndIf
    \If{There is an idle processor}
      \If{There is a queued task}
        \If{backlog $\ge 1$}
          \State schedule tasks in serial on any idle processors
        \EndIf
        \If{backlog $< 1$}
          \State schedule a task in parallel, scheduling as
          balancedly as possible
        \EndIf
      \EndIf
      \If{There is no queued task \textbf{and} there is a serial
      task that could be cancelled and then redistributed that
    would result in all cups that are getting the redistribution
  stuff end up with less work than the thing that was cancelled
from}
        \State Cancel the serial task and reschedule as specified
      \EndIf
    \EndIf
    \EndProcedure
  \end{algorithmic}
\end{algorithm}

\begin{claim}
  Alg X is good.
\end{claim}
\begin{proof}
  This seems basically impossible to prove, it's a super
  complicated algorithm with so much branching. 
\end{proof}

\section{some more strategies}

\begin{itemize}
  \item randomized (smoothed analysis)
  \item look at discrete version?
\end{itemize}


\begin{algorithm}
  \caption{Randomized alg}
  \begin{algorithmic}[1]
    \State This procedure is continuously running.
    \Procedure{randomizedStrategy}{}
    \State 
    \If{backlog would be made smaller by cancelling everything
    and swapping the mode, and scheduling everything according ot
  the new mode}
      \State do it!
    \EndIf
    \EndProcedure
  \end{algorithmic}
\end{algorithm}

\begin{algorithm}
  \caption{Alg binary}
  \begin{algorithmic}[1]
    \State This procedure is continuously running.
    \Procedure{binary}{mode}
    \State when you get a new task schedule it to minimize
    backlog in the current mode (mode is serial or parallel)
    \If{backlog would be made smaller by cancelling everything
    and swapping the mode, and scheduling everything according ot
  the new mode}
      \State do it!
    \EndIf
    \EndProcedure
  \end{algorithmic}
\end{algorithm}

\end{document}

\section{Dependencies}
\label{sec:dependencies}

\subsection{Unknown Dependencies}
We now give an on-line scheduling algorithm in
Algorithm~\ref{alg:chunk}, which we call \defn{CHUNK}, that is
$O(\sqrt{p})$-competitive with the optimal off-line scheduling
algorithm for the problem where there are unknown dependencies
among the tasks.
Recalling the $\Omega(\sqrt{p})$ lower bound on such an
algorithm's competitive ratio, we have that our scheduler is
asymptotically optimal in this situation.

\begin{algorithm}
  \caption{CHUNK}
  \label{alg:chunk}
  \begin{algorithmic}
    \State $P_i \gets \bigcup_{j=(i-1)\sqrt{p}/2+1}^{i\sqrt{p}/2} \{\rho_j\}$ for $i \in [\sqrt{p}/2+1]$
    \State
    \Comment Group $\rho_j$'s into chunks $P_i$ of $\sqrt{p}/2$ processors
    \State $A = \{P_1,P_2, \ldots, P_{\sqrt{p}/2} \} $
    \State $B = P_{\sqrt{p}/2 + 1}$
    \State $A$-verge-time: time where no chunks in $A$ are doing work
    \State $B$-verge-time: time where no processors in $B$ are doing work

    \State Let $\mathcal{T}$ be the set of arrived tasks that haven't been started yet.
    \If{time is $A$-verge-time}
    \State Let $\mathcal{T}_A = \{ \tau \in \mathcal{T} | \pi(\tau) < \sqrt{p}/2 \cdot \sigma(\tau) \}.$
      \State For each $\tau \in \mathcal{T}_A$, let $\tau'$ be
      \defn{chunked} $\tau$, a task that must run in serial that
      has work $\pi(\tau)/(\sqrt{p}/2)$.
      \State Schedule the chunked tasks $\{\tau' | \tau \in
      \mathcal{T}_A\}$ each on a single chunk in $A$, scheduling with
      sequential bin packing (on sorted tasks), which is
      constant-competitive with what OPT would do.
    \EndIf
    \If{time is $B$-verge-time}
      \State Let $\mathcal{T}_B = \{ \tau\in \mathcal{T}  | \pi(\tau) \ge \sqrt{p}/2 \cdot \sigma(\tau) \}.$
      \State Schedule the tasks $\mathcal{T}_B$ on a processor in
      $B$, scheduling with sequential bin packing (on sorted
      tasks), which is constant-competitive with what OPT would do.
    \EndIf
  \end{algorithmic}
\end{algorithm}

We will show that CHUNK with $p$ processors is $O(1)$-competitive
with OPT on $\sqrt{p}/2$ processors, which we refer to as
\defn{SmallOPT}.

We now claim that the awake time of CHUNK is constant-competitive
with the awake time of SmallOPT. We define the \defn{CHUNK-$X$
awake time}, for $X=A,B$, to be the amount of time that $X$ has
unfinished tasks; we will show that CHUNK-$X$ awake time is
constant-competitive with SmallOPT's awake time for $X=A,B$.

First consider $X=B$. Recall that $B$ gets tasks with cost ratio
at least $\sqrt{p}/2$. SmallOPT must run these tasks in serial,
and CHUNK chooses to run these tasks in serial as well (on a
processor in $B$). Without loss of generality we restrict to
considering TAPs that consist only of tasks with cost ratio at
least $\sqrt{p}/2$; considering TAPs with other tasks increases
SmallOPT's awake time without affecting the CHUNK-$B$ awake time.
By Theorem~\ref{thm:constant_competitive} and the analysis of
bin-packing we have that the CHUNK-$B$ awake time is
$O(1)$-competitive with SmallOPT's awake time.

Now consider $X=A$. Recall that $A$ gets tasks with cost ratio
less than $\sqrt{p}/2$. SmallOPT can run these tasks in serial or
in parallel. CHUNK runs these tasks in parallel on a single
chunk. For a task $\tau$, the time it takes to run on a single
chunk is $\pi(\tau)/(\sqrt{p}/2) < \sigma(\tau)$. Without loss of
generality we restrict to considering TAPs that consist only of
tasks with cost ratio less than $\sqrt{p}/2$; considering TAPs
with other tasks increases SmallOPT's awake time without
affecting the CHUNK-$A$ awake time. We can augment SmallOPT by
considering any task run in parallel by SmallOPT to just be the
work on a processor assigned the most of its work. Then its
serial vs serial, and by Theorem~\ref{thm:constant_competitive}
and the analysis of bin-packing we have that the CHUNK-$A$ awake
time is $O(1)$-competitive with SmallOPT's awake time.

But of course CHUNK's awake time is at most the sum of CHUNK-$A$
awake time and CHUNK-$B$ awake time. So CHUNK's awake time is
constant-competitive with SmallOPT's awake time.

SmallOPT is $O(\sqrt{p})$-competitive with OPT since SmallOPT
could just use $2\sqrt{p}$ steps to simulate a step of OPT. Thus,
because CHUNK is constant-competitive with SmallOPT, CHUNK is
$O(\sqrt{p})$-competitive with OPT. 

\subsection{Known Dependencies}
Now we consider the off-line version of the dependency version of
the scheduling problem. In particular, in this version of the
problem the scheduler has full knowledge of all the specs of all
the tasks and all the dependencies.

We give an algorithm to constant-approximate the optimal off-line
scheduler in reasonable running time.


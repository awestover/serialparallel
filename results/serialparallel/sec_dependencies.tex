\section{Dependencies}
\label{sec:dependencies}

\subsection{Unknown Dependencies}
We now give an on-line scheduling algorithm in
Algorithm~\ref{alg:chunk}, which we call \defn{CHUNK}, that is
$O(\sqrt{p})$-competitive with the optimal off-line scheduling
algorithm for the problem where there are unknown dependencies
among the tasks.
Recalling the $\Omega(\sqrt{p})$ lower bound on such an
algorithm's competitive ratio, we have that our scheduler is
asymptotically optimal in this situation.

\begin{algorithm}
  \caption{CHUNK}
  \label{alg:chunk}
  \begin{algorithmic}
    \State $P_i \gets \{\rho_{i\sqrt{p}}, \rho_{i\sqrt{p}+1}, \ldots, \rho_{(i+1)\sqrt{p}-1}\}$ for $i \in [\sqrt{p}/2]$
    \State
    \Comment Group $\rho_j$'s into chunks $P_i$ of $\sqrt{p}$ processors
    \State $A = \{P_1,P_2, \ldots, P_{\sqrt{p}/2} \} $
    \State $B = \{\rho_{\sqrt{p}/2 + 1},\rho_{\sqrt{p}/2 + 2}, \ldots, \rho_{\sqrt{p}} \} $

    \If{There is an unscheduled task $\tau$}
      \If{$\pi(\tau) \le \sqrt{p} \cdot \sigma(\tau)$}
        \State Schedule $\tau$ in parallel, distributing work equally, on whichever chunk in $A$ has the least work
      \Else
        \State Schedule $\tau$ in serial on whichever processor in $B$ has the least work
      \EndIf
    \EndIf
  \end{algorithmic}
\end{algorithm}

We will show that CHUNK with $p$ processors is $O(1)$-competitive
with OPT on $\sqrt{p}/2$ processors, which we refer to as
\defn{SmallOPT}.

{\color{red}
Call a time step \enquote{super productive} if either: all of processors in A are working or all of processors in B are working.
CHUNK's total amount of super productive time is constant competitive with OPT's total amount of time.
So we can just focus on time steps that aren't super productive for CHUNK. 
But in those time steps, every single job that's present is
running, and it's running at a speed so that the total time it
takes will be competitive with the total time that the same job
runs for in OPT.
So it seems like (and this is still a sketchy argument that I
don't have ironed out, and may be wrong) we should be
constant-competitive with SmallOPT.
}

SmallOPT is $O(\sqrt{p})$-competitive with OPT since SmallOPT
could just use $\sqrt{p}$ steps to simulate a step of OPT. Thus,
because CHUNK is constant-competitive with SmallOPT, CHUNK is
$O(\sqrt{p})$-competitive with OPT. 

\subsection{Known Dependencies}
We might hope to do better if we have more data; the lower bound
construction is heavily dependent on the structure of the
dependencies being invisible.



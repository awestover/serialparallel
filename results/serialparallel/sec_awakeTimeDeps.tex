\section{Minimizing Awake Time on DTAPs} 
\label{sec:awaketimeDeps}

\subsection{Off-line $\Omega(\sqrt{p})$ Lower Bound}
In the unknown dependencies it is impossible to get an
$O(1)$-competitive algorithm with the optimal algorithm that
knows the dependencies.
In particular, we give a DTAP with $n=p$ tasks, that can be
processed in time $O(1)$ by an algorithm that knows the
dependencies, but that takes time
$\Omega(\sqrt{n})$ to be processed in the worst
case by any deterministic algorithm that does not know the
dependencies.
In fact, we can show any deterministic or randomized algorithm
takes time $\Omega(\sqrt{n})$ to process this DTAP with high
probability in $n$.

We now construct and analyze the DTAP. We will make $n=p$ tasks.
The DTAP consists of $\sqrt{p}$ \defn{levels}. On each level of
the DTAP the tasks are $(1,\sqrt{p})\times \sqrt{p}$. One of
these tasks spawns the $\sqrt{p}$ tasks at the next level of
recursion upon completion.

A scheduler that knew the dependencies would first run all the
spawning tasks using their parallel implementations. Doing so the
emptier could unlock all the tasks in time $\sqrt{p} \sqrt{p} / p
= 1$. Then there are less than $p$ tasks, so the scheduler can
schedule them all with their serial implementations, and finishes
$1$ unit of time later. In total this gives awake time $2$.
But if the scheduler does not know which are the spawning tasks,
then it can't immediately do them. In the worst case such a
scheduler will take $1$ unit of time to uncover the dependencies:
if the scheduler runs any task in serial it could take $1$ unit
of time, and if all tasks are run in parallel it could take
$\sqrt{p}\sqrt{p} / p = 1$ unit of time.
Taking $1$ unit of time on each of the $\sqrt{p}$ levels means
that this scheduler has awake time at least $\sqrt{p} = \sqrt{n}$.

By doing tasks in parallel until the spawning task is uncovered a
randomized emptier can hope to do slightly better. But it turns
out not better asymptotically.
With high probability in $n$ the scheduler takes time at least
$1/2$ on at least $1/4$ of the levels.
Hence the scheduler takes time at least $1/8 \sqrt{n}$ with high
probability in $n$, as desired.

\subsection{Off-line $O(\sqrt{p})$-competitive algorithm}
We now give an on-line scheduling algorithm in
Algorithm~\ref{alg:aug}, which we call \defn{AUG}, that is
$O(\sqrt{p})$-competitive with the optimal off-line scheduling
algorithm for the problem where there are unknown dependencies
among the tasks.
Recalling the $\Omega(\sqrt{p})$ lower bound on such an
algorithm's competitive ratio, we have that our scheduler is
asymptotically optimal in this situation.

\begin{algorithm}
  \caption{AUG}
  \label{alg:aug}
  \begin{algorithmic}
    \State $A = \{\rho_1, \rho_2, \ldots, \rho_{p/2} \}$
    \State $B = \{\rho_{1+p/2}, \rho_{2+p/2}, \ldots, \rho_{p} \}$
    \State
    \State $\mathcal{T} \gets$ arrived tasks that haven't been scheduled yet
    \State $\mathcal{T}_A \gets \{ \tau \in \mathcal{T} | \pi(\tau) \le \sqrt{p} \cdot \sigma(\tau) \}.$
    \State $\mathcal{T}_B \gets \{ \tau\in \mathcal{T}  | \pi(\tau) > \sqrt{p} \cdot \sigma(\tau) \}.$
    \State
    
    \If{$B$ has no tasks running}
      \State Sequentially greedily schedule all tasks $\tau \in \mathcal{T}_B$ on processors in $B$
    \EndIf
    \State Schedule all tasks $\tau \in \mathcal{T}_A$ in parallel on all processors in $A$

  \end{algorithmic}
\end{algorithm}

We will show that AUG with $p$ processors is $O(1)$-competitive
with OPT on $\sqrt{p}$ processors, which we refer to as
\defn{SmallOPT}. We define the \defn{AUG-$X$
awake time}, for $X=A,B$, to be the amount of time that $X$ has
unfinished tasks; we will show that AUG-$X$ awake time is
constant-competitive with SmallOPT's awake time for $X=A,B$.

First consider $X=A$. Recall that $A$ gets tasks with cost ratio
at most $\sqrt{p}$. SmallOPT can run these tasks in serial or in
parallel. AUG runs these tasks in parallel on all processors in
$A$. For a task $\tau$, the time it takes AUG to run the task is
$\pi(\tau)/(p/2) < \sigma(\tau)/ (\sqrt{p}/2)$. The AUG-$A$ awake
time is 
$$\sum_{\tau \in \mathcal{T}_A} \frac{\pi(\tau)}{p/2} \le \sum_{\tau \in \mathcal{T}_A} \frac{\sigma(\tau)}{\sqrt{p}/2}$$
where $\mathcal{T}_A$ denotes the set of tasks with cost ratio at most $\sqrt{p}$.
On the other hand, the awake time of SmallOPT is trivially at least 
$$\sum_{\tau} \frac{\sigma(\tau)}{\sqrt{p}}.$$
So AUG-$A$ awake time is constant-competitive with SmallOPT's awake time.

Now consider $X=B$. Recall that $B$ gets tasks with cost ratio
larger than $\sqrt{p}$. SmallOPT must run these tasks in serial,
and CHUNK chooses to run these tasks in serial as well, in
particular on processors in $B$. Without loss of generality we
restrict to considering DTAPs that consist only of tasks with
cost ratio larger than least $\sqrt{p}$; considering DTAPs with
other tasks increases SmallOPT's awake time without affecting
CHUNK-$B$ awake time. We claim that by scheduling tasks greedily
in batches AUG-$B$ awake time is $4$-competitive with SmallOPT's
awake time. We define a \defn{$B$ start time} to be a time when
$B$ has no work and there are tasks that have already arrived
with cost ratio larger than $\sqrt{p}$ that $B$ is about to
schedule. We define a \defn{$B$ finish time} to be a time when
$B$ finishes a batch of tasks. Note that $B$ finish times and $B$
start times certainly can coincide. 
{\color{red} % this next part is baloney
Let $T_t^{start}$ denote the
difference between the $t$-th finish time and the $t$-th start
time, i.e. the amount of time that greedy bin-packing takes to
process the $t$-th batch of tasks. Let $T_t^{during}$ denote the
amount of time that greedy bin-packing takes to process the tasks
that arrived between the $t$-th start time and the $t$-th finish
time. Clearly AUG-$B$ awake time is at most 
$$\sum_{t} \left(T_t^{start} + T_t^{during}\right).$$
On the other hand, greedy bin-packing is $2$-competitive with OPT
for serial batches, so SmallOPT's awake time is at least 
$$\sum_{t} \max(T_t^{start}/2, T_t^{during}/2).$$
Hence AUG-$B$ awake time is $4$-competitive with SmallOPT's awake time.
}

But of course AUG's awake time is at most the sum of AUG-$A$
awake time and AUG-$B$ awake time. So AUG's awake time is
constant-competitive with SmallOPT's awake time.

SmallOPT is $(\sqrt{p})$-competitive with OPT since SmallOPT
could just use $\sqrt{p}$ steps to simulate a step of OPT. Thus,
because AUG is constant-competitive with SmallOPT, AUG is
$O(\sqrt{p})$-competitive with OPT. 

\subsection{On-line Scheduler}
Now we consider the off-line version of the dependency version of
the scheduling problem. In particular, in this version of the
problem the scheduler has full knowledge of all the specs of all
the tasks and all the dependencies.

We give an algorithm to constant-approximate the optimal off-line
scheduler in reasonable running time.


\documentclass[twocolumn]{article}[10pt]
\usepackage[left=1in, right=1in, top=1in, bottom=1in]{geometry}
% \usepackage[subtle]{savetrees} 

\usepackage{amsthm}
\usepackage{amssymb}
\usepackage{amsmath}
\usepackage{mathtools}
\usepackage{xcolor}
\usepackage{xspace}
\usepackage{csquotes} 

\usepackage{fancyhdr}
\pagestyle{fancy}
\fancyhead{}
\fancyfoot{}
\fancyfoot[R]{\thepage}
\renewcommand{\headrulewidth}{0pt}

\usepackage{hyperref}

\newcommand{\defn}[1]{{\textit{\textbf{\boldmath #1}}}\xspace}
\renewcommand{\paragraph}[1]{\vspace{0.09in}\noindent{\bf \boldmath #1.}} 
\newcommand{\todo}[1]{{\color{red}\textbf{TODO:} #1}}

\DeclareMathOperator{\E}{\mathbb{E}}
\DeclareMathOperator{\Var}{\text{Var}}
\DeclareMathOperator{\img}{Img}
\DeclareMathOperator{\polylog}{\text{polylog}}
\newcommand{\defeq}{\vcentcolon=}
\newcommand{\eqdef}{=\vcentcolon}

\DeclareMathOperator{\work}{\text{work}}
\newcommand{\PR}[1]{\text{PR}( #1 )}

\newcommand{\floor}[1]{\left\lfloor #1 \right\rfloor}
\newcommand{\ceil}[1]{\left\lceil #1 \right\rceil}
\newcommand{\paren}[1]{\left( #1 \right)}

\usepackage{algorithm}
\usepackage[noend]{algpseudocode} % adding [noend] deletes the end while and stuff

\usepackage[capitalise,nameinlink,noabbrev]{cleveref}
\crefname{equation}{}{} % cref{eq:blah} only does (1) instead of Equation (1)
\crefname{enumi}{Step}{} % cref{eq:blah} only does (1) instead of Equation (1)

\newtheorem{fact}{Fact}
\newtheorem{definition}{Definition}
\newtheorem{remark}{Remark}
\newtheorem{proposition}{Proposition}
\newtheorem{claim}{Claim}
\newtheorem{conjecture}{Conjecture}
\newtheorem{lemma}{Lemma}
\newtheorem{corollary}{Corollary}
\newtheorem{theorem}{Theorem}

\author{Alek Westover}
\title{Serial-Parallel Scheduling Problem}

\begin{document}
\maketitle

\begin{abstract}
There are many problems for which the best parallel algorithms
have larger cost than the best serial algorithms, i.e. are not
work-efficient. We consider a scheduler that is receiving many
tasks with serial and parallel implementations that have
potentially different costs. The scheduler can choose whether to
run each task in serial or in parallel, and aims to either
minimize total awake time, i.e. the amount of time that the
scheduler has unfinished tasks. 

We show that, very surprisingly, the off-line problem can
essentially be reduced to the case where all tasks are available
from the start, a setting in which the off-line and on-line
algorithms are of course the same. In particular, we give a
simple deterministic $2$-competitive off-line scheduling
algorithm, that does not even need to use preemption! The
$2$-competitive algorithm relies on solving a special case of the
off-line problem where all tasks arrive at the same time. We give
a $3$-approximation algorithm for the single-arrival-time
off-line problem with running time $O(n)$.

We also prove several impossibility results.
We show that no deterministic scheduler can have a
competitive ratio smaller than $1.25$ in general.
Even with randomization, we show that for any randomized
algorithm there is some input on which the algorithm achieves
competitive ratio at least $1.0625$ with high probability.

We also consider a generalization of the scheduling problem where
tasks can have dependencies. Here we show XXX.
\end{abstract}

\section{Introduction}
\label{sec:intro}
\subsection{Problem Specification}
A parallel algorithm is said to be \defn{work-efficient} if the
work of the parallel algorithm is the same as the work of a
serial algorithm for the same problem. Most implementations of
parallel algorithms are not work-efficient, often having work
that is a constant factor greater, or even asymptotically
greater, than the work of the serial algorithm for the problem.

In the \defn{Serial-Parallel Scheduling Problem} we have to
perform $n$ tasks $\tau_1, \ldots, \tau_n$ ($n$ unknown ahead of
time). We have $p$ processors $\rho_1, \ldots, \rho_p$. Each task
$\tau_i$ has a parallel implementation with work $\pi(\tau_i)$ and a
serial implementation with work $\sigma(\tau_i)$. The tasks will become
available at some times $t(\tau_1), \ldots, t(\tau_n)$. 
The set of tasks with their associated parallel and serial
implementation's works and with their associated arrival times is
called a \defn{task arrival plan} or \defn{TAP} for short.

The scheduler maintains a set of \defn{ready} tasks, which are
tasks that have become available but are not currently being run
on any processor. At time $t(\tau_i)$ task $\tau_i$ is added to
the set of ready tasks. At any time the scheduler can decide to
schedule some (not already running) ready task, and can choose
whether to run the task in serial, in which case the scheduler
must choose a single processor to run the task on, or in
parallel, in which case the scheduler can distribute the task's
work arbitrarily among the processors. Intuitively, if there are
many ready tasks then the scheduler should run the serial
implementations of the tasks because the scheduler can achieve
parallelism across the tasks. On the other hand, if there are not
very many ready tasks it is probably better for the scheduler to
run the parallel versions of the tasks --- even though they are
possibly not work efficient, i.e. $\pi(\tau) > \sigma(\tau)$ ---
because by so doing at least the scheduler can achieve
parallelism within tasks.

We also consider a generalization of the Serial-Parallel
Scheduling Problem to the case where the task arrival times are
not fixed, but rather some tasks may become available only after
the completion of other tasks. Put another way, we consider a
version of the problem where the tasks have dependencies. We
refer to the set of tasks along with their associated parallel
and serial implementation's works and their associated arrival
times or dependency (i.e. what task they spawn after) as a
\defn{DTAP} (D stands for dependency).

Let the \defn{awake time} of the scheduler be the duration of
time over which the scheduler has unfinished tasks. One natural
goal for the scheduler is to minimize awake time. We measure how
well the scheduler is able to minimize its awake time by
comparing its awake time to the awake time of the optimal
off-line strategy, which we will denote OPT. Note that OPT is
able to see the whole sequence of tasks in advance. The
\defn{competitive ratio} of a scheduler is the ratio of its awake
time to the awake time of OPT on the same input.

% Another interesting metric to consider is \defn{mean response
% time}, the average over the tasks of the time between when the
% task arrives and when the task is finished. Again we can consider
% the competitive ratio of a scheduler relative to this metric.

\subsection{Notation}
We use $[n]$ to denote $\{1,2,\ldots, n\}.$

\subsection{Problem Motivation}
Data-centers often get heterogeneous tasks. Being able to schedule
them efficiently is a fundamental problem. 

In the CILK programming language whenever a function is called we
could let the scheduler choose between a serial and a parallel
implementation of a task. 

\subsection{Related Work}
Shortest Job First (SJF) is a pretty common idea for minimum response time. 
An algorithm called Shortest Remaining Processing Time (SRPT),
and its variants are often useful for minimizing metrics like mean
response time.

In \cite{bb20}, Berg et al study a related problem:
many heterogeneous tasks come in, some which are elastic and
exhibit perfect linear scaling of performance and some which are
inelastic which must be run on a single processor, according to
some stochastic assumptions, and they aim to minimize mean
response time. They show that for some parameter settings the
strategy \enquote{Inelastic First} is optimal.

In \cite{is16} Im et al exhibit an algorithm keeps the average
flow time small. 

In \cite{ga12} Gupta et al prove some impossibility results about
a problem somewhat similar to our problem.

Clearly related problems are widely studied.
Our problem is novel however, and interesting.

\subsection{Results}
\todo{copy abstract here and reword it}

\section{An Algorithm for Optimizing Awake Time} 
\label{sec:awaketime}
In this section we give a simple deterministic scheduling
algorithm --- that does not use preemption --- for minimizing
awake time. We show that our algorithm is $2$-competitive
with OPT on all TAPs. Our algorithm is theoretically interesting,
but not efficient. We also give an efficient algorithm for
$2$-approximating our algorithm.

Note that without loss of generality we may consider TAPs where
the cost ratio $\pi(\tau)/\sigma(\tau) \in [1,p]$ for all tasks
$\tau$; if $\pi(\tau)/\sigma(\tau) < 1$ then the scheduler
clearly should never run $\tau$ in serial so we can replace the
serial implementation with the parallel implementation to get
cost ratio $1$, similarly, if $\pi(\tau)/\sigma(\tau) > p$ then
the scheduler should never run $\tau$ in parallel and we can
replace the parallel implementation with the serial
implementation to get cost ratio $p$.

We say that a time is a \defn{verge} time for our algorithm if at
this time no processors have work assigned to them, and there is at
least one ready task.

We propose Algorithm~\ref{alg:bat}, which we call \defn{BAT}
(BAT is short for \enquote{batch}), as a scheduling algorithm.

\begin{algorithm}
  \caption{BAT}
  \label{alg:bat}
  \begin{algorithmic}
    \State Let $\oracle_R$ be an algorithm that is $R$-competitive with FROST 
    \While{True}
      \If{verge time}
        \State schedule tasks as directed by $\oracle_R$
      \EndIf
    \EndWhile
  \end{algorithmic}
\end{algorithm}

A \defn{single-time-TAP} is a TAP where all tasks arrive at a
single time. FROST is an algorithm that yields a schedule
achieving minimal awake time, i.e. the same awake time as OPT, on
single-time-TAPs. In Lemma~\ref{lem:frosting} we establish that
Algorithm~\ref{alg:frost} implements FROST, hence showing
that BAT can actually be computed. Algorithm~\ref{alg:frost}
exactly computes FROST, but is very slow. We also give an
algorithm THRESH in Algorithm~\ref{alg:thresh} for
$2$-approximating FROST in linear time. We remark that it is not
obvious that an oracle for OPT can be computed in finite time,
even in this special case: there are an uncountably infinite
number of possible scheduling strategies that OPT can choose
from. Nevertheless, we show how to consider a finite search space
that a search can actually be performed on, at least in the
special case of single-time-TAPs. Before considering FROST, and
approximations to FROST, we analyze the competitive ratio of BAT
assuming that we have $\oracle_R$, an algorithm that is
$R$-competitive with FROST for some $R \le O(1)$; $\oracle_1$
corresponds to FROST while $\oracle_2$ corresponds to THRESH.

Consider a TAP $\mathcal{T}$. Let $\ell$ be the number of verge
times for $\mathcal{T}$; note that $\ell \le n$ which in
particular is finite. Let $t_i$ be the $i$-th time that is a
verge time, let $q_i$ be the number of ready tasks for BAT at
time $t_i$. Let $T^{ALG}(q_1, \ldots, q_{\ell'})$ denote the
awake time of a scheduling algorithm ALG on the truncation of the
TAP $\mathcal{T}$ that only consists of tasks arriving at times
before $t_{\ell'}$.

By definition of $\oracle_R$ we have that $\oracle_R$ is
$R$-competitive with OPT on single-time-TAPs, i.e.
\begin{equation}
  \label{eq:same_single}
  T^{BAT}(q) \le R \cdot T^{OPT}(q).
\end{equation}

An \defn{ALG-gap} is an interval of time $I$ of non-zero length where for
all times in the interior of $I$ ALG has completed every
task that has arrived thus far. Additionally, for an interval to
be an ALG-gap the interval must contain no other intervals which
are also ALG-gaps (i.e. it is a \enquote{maximal} interval
satisfying our conditions).
We say that a TAP is \defn{ALG-gap-free} if it contains no ALG-gaps.

Now we prove an obvious property of OPT.
\begin{claim}
  \label{clm:OPT_finishes_first}
  If there is a scheduling algorithm ALG that completes all tasks by
  time $t_*$ then OPT finishes all tasks by time $t_*$.
\end{claim}
\begin{proof}
  Say that ALG completes all tasks by time $t_*$. Let $t_0 < t_*$
  be the most recent time that OPT has completed all tasks that
  arrive before time $t_0$. If OPT has not finished all tasks by
  time $t_*$ then it was acting sub-optimally, as it could steal
  the strategy that ALG used on $[t_0, t_*]$ to achieve lower
  awake time. In particular, for any tasks that arrive in $[t_0,
  t_*]$ OPT could schedule them as ALG schedules them. 
\end{proof}
As an immediate consequence of Claim~\ref{clm:OPT_finishes_first}
we have that any ALG-gap is a subset of an OPT-gap.

Decomposing TAPs into gap-free subsets of the TAP is very useful.
Part of the reason for this is the following fact:
\begin{claim}
  \label{clm:just_consider_gapless}
  If an algorithm ALG achieves competitive ratio $r$ on
  ALG-gap-free TAPs, then ALG achieves 
  competitive ratio $r$ on arbitrary TAPs.
\end{claim}
\begin{proof}
  We partition the tasks based on arrival time, splitting the
  tasks on the ALG-gaps. That is, we split the tasks into groups
  so that two tasks $\tau_i, \tau_j$ are in the same group if and
  only if there are no ALG-gaps in between the arrival times of
  $\tau_i$ and $\tau_j$.
  We can define an interval of time $I_i$ for each of these
  ALG-gap-free subsets of the TAP, where $I_i$ is defined so that
  all tasks in the $i$-th group start and finish at times
  contained in the interval $I_i$.

  Let $T_{I_i}^{OPT}$ and $T_{I_i}^{ALG}$ denote the awake time
  of OPT and ALG on interval $I_i$. Because $I_i$ contains no
  ALG-gaps we have $T^{ALG} = \sum_{i} T^{ALG}_{I_i}$.
  Further, recall that by Claim~\ref{clm:OPT_finishes_first} any
  ALG-gap is also an OPT-gap, so
  $T^{OPT} = \sum_{i} T_{I_i}^{OPT}$. 
  Hence from our assumption that ALG is $r$-competitive on
  gap-free TAPs, such as the subset of the TAP on the interval
  $I_i$, we have $T_{I_i}^{ALG} \le r\cdot T_{I_i}^{OPT}$ for
  all $i$. Summing we get $T^{ALG} \le r\cdot T^{OPT}$, as desired.
  
\end{proof}

By Claim~\ref{clm:just_consider_gapless}, in order to bound BAT's
competitive ratio, it suffices to consider TAPs
without BAT-gaps. Note however that a TAP without
BAT-gaps could still have OPT-gaps.

We conclude our analysis of the competitive ratio of BAT in
Theorem~\ref{thm:constant_competitive} with an inductive argument on
the number of OPT-gaps in the TAP.
First we establish the base case for the argument in
Proposition~\ref{prop:no_optgaps}: we consider
BAT's competitive ratio on an OPT-gap-free TAP.  

\begin{proposition}
  \label{prop:no_optgaps}
  BAT is $(R+1)$-competitive on OPT-gap-free TAPs.
\end{proposition}
\begin{proof}
  For an OPT-gap-free TAP we must have
  \begin{equation}
    \label{eq:opt_isnt_so_much_better}
    T^{OPT}(q_1, \ldots, q_{\ell}) \ge T^{BAT}(q_1, \ldots, q_{\ell-1}).
  \end{equation}
  Because BAT finishes all $q_{i}$ tasks that arrive at time $t_i$
  by time $t_{i+1}$ we can actually always decompose
  $T^{BAT}(q_1, \ldots, q_\ell)$ as 
  \begin{equation}
    \label{eq:decomposeBAT}
    T^{BAT}(q_1, \ldots, q_\ell) = \sum_{i=1}^\ell T^{BAT}(q_i).
  \end{equation}
  By Equation~\eqref{eq:decomposeBAT}, and
  Equation~\eqref{eq:same_single} we thus have 
  \begin{equation}
    \label{eq:decompose_rearanged}
    T^{BAT}(q_1, \ldots, q_\ell) \le T^{BAT}(q_1, \ldots, q_{\ell-1}) + R \cdot T^{OPT}(q_\ell).
  \end{equation}

  Hence by Equation~\eqref{eq:opt_isnt_so_much_better} and
  Equation~\eqref{eq:decompose_rearanged} we have
  \begin{align*}
    T^{BAT}(q_1, \ldots, q_\ell) &\le T^{OPT}(q_1, \ldots, q_\ell) + R\cdot T^{OPT}(q_\ell)\\
                                 &\le (R+1)T^{OPT}(q_1, \ldots, q_\ell),
  \end{align*}
  as desired.
\end{proof}

\begin{theorem}
  \label{thm:constant_competitive}
  BAT is $(R+1)$-competitive.
\end{theorem}
\begin{proof}
  The proof is by strong induction on the number of OPT-gaps. 
  The base case of our induction is established in
  Proposition~\ref{prop:no_optgaps}, which says that if there are $0$
  OPT-gaps then BAT is $2$-competitive. 

  Consider a TAP that has more than $0$ OPT gaps;
  say that its first OPT-gap starts at time $t_*$.
  Let $j$ be the largest index such that verge time $t_j < t_*$.

  Using our inductive hypothesis we have:
  \begin{align*}
  &T^{OPT}(q_1, \ldots, q_\ell) \\
  &\ge T^{OPT}(q_1, \ldots, q_j) + T^{OPT}(q_{j+1}, \ldots, q_{\ell})\\
  &\ge \frac{1}{R+1}\paren{T^{BAT}(q_1, \ldots, q_j) + T^{BAT}(q_{j+1}, \ldots, q_{\ell})}\\
  &=\frac{1}{R+1} T^{BAT}(q_1, \ldots, q_\ell).
  \end{align*}

  Rearanging we get the desired bound:
  $$T^{BAT}(q_1, \ldots, q_\ell) \le (R+1) T^{OPT}(q_1, \ldots, q_\ell).$$
\end{proof}

Now we analyze Algorithm~\ref{alg:frost}, which we call
FROST, or $\oracle_1$. 

\begin{algorithm}
  \caption{FROST (i.e. $\oracle_1$)}
  \label{alg:frost}
  \begin{algorithmic}
    \State \textbf{Input:} tasks $\tau_1,\ldots, \tau_n$ all with $t(\tau_i) = 0$
    \State \textbf{Output:} a way to schedule the tasks to
    processors $\rho_1, \ldots, \rho_p$ that achieves minimal awake time
    \State 
    \State $\text{minAwakeTime} \gets \infty$
    \State $\text{bestSchedule} \gets $ schedule everything in serial on $\rho_1$
    \For{$I \in \{0,1\}^n$} 
      \State $x \gets \sum_{i=1}^n I_i$
      \For{$J \in \{1, \ldots, p\}^x$}
        \State $j \gets 0$
        \For{$i \in \{1,2,\ldots, n\}$}
          \If{$I_i=1$}
            \State $j \gets j+1$
            \State schedule task $\tau_i$ in serial on $\rho_{J_j}$
          \EndIf
        \EndFor
        \State $m \gets \max_{\rho_i}(\work(\rho_i))$
        \State $w \gets \sum_{\rho_i} \paren{m - \work(\rho_i)}$
        \State $f \gets \sum_{\rho_i} (1-I_i)\pi(\rho_i)$
        \If{$f \ge w$}
          \State make $\rho_i$ have work $m + (f-w)/p$
        \Else 
          \State distribute $f$ units of work arbitrarily 
          \State among $\rho_i$ without increasing awake time
        \EndIf
        \If{$\text{awakeTime}(I, J) \le \text{minAwakeTime}$}
          \State{$\text{minAwakeTime} \gets \text{awakeTime}(I, J)$}
          \State{$\text{bestSchedule} \gets \text{schedule}(I, J)$}
        \EndIf
      \EndFor
    \EndFor
  \end{algorithmic}
\end{algorithm}

The key insight to decrease the search space to be finite is to
notice that for any method of distributing whichever tasks are
chosen to run in serial, the parallel tasks may as well be
redistributed afterwords, so long as doing so either results in
not increasing the awake time, or results in all tasks having
identical amounts of work.
We can thus do a brute-force search over all the ways to assign some
tasks to run in serial and to run on specific processors, and
then put the parallel tasks on top \enquote{like frosting on a cake}. 

We remark that the running time of FROST for $n$ tasks is 
larger than $\Omega(p^n)$, which is extremely large. Nevertheless the
existence of the algorithm is interesting. Later we give an
algorithm for $\oracle_2$ that has much more reasonable running
time.

We now prove that FROST actually does compute a schedule with
awake time the same as that of OPT.
\begin{lemma}[Frosting Lemma]
  \label{lem:frosting} 
  FROST is an oracle for OPT on TAPs where all tasks arrive at a single time.
\end{lemma}
\begin{proof}
  Consider the configuration that OPT chooses. FROST considers
  a configuration of tasks with the same assignment of serial
  tasks at some point, because FROST brute force searches
  through all of these. For this configuration it is clearly
  impossible to achieve lower awake time than by spreading the
  parallel tasks in the frosting method, hence OPT's awake time
  is at least that of FROST.
\end{proof}

We now consider how to
more efficiently compute the best schedule for the case where all
tasks arrive at the start. In fact, we consider making an
algorithm to get a constant approximation of this, which will
still give an algorithm with constant competitive ratio.

We propose Algorithm~\ref{alg:thresh}, which we call THRESH, as a
simple way to $2$-approximate FROST.

\begin{algorithm}
  \caption{THRESH}
  \label{alg:thresh}
  \begin{algorithmic}
    \State \textbf{Input:} tasks $\tau_1,\ldots, \tau_n$ with
    $\sigma(\tau_1) \ge \sigma(\tau_2)\ge \cdots \ge
    \sigma(\tau_n)$ all with $t(\tau_i) = 0$
    \State \textbf{Output:} a way to schedule the tasks to
    processors $\rho_1, \ldots, \rho_p$ that achieves awake time
    at most twice the optimal awake time.
    \State
    \State $w_\sigma \gets \sum_i \sigma(\tau_i)$
    \State $w_\pi \gets 0$
    \State $a \gets 0$, $i_* = 0$
    \For{$i \in [n]$}
      \State $w_\sigma \gets w_\sigma - \sigma(\tau_i)$
      \State $w_\pi \gets w_\pi + \pi(\tau_i)$
      \If{$i < n$}
        \State $a_0 = w_\sigma / p + \sigma(\tau_{i+1}) + w_\pi/p$
      \Else
        \State $a_0 = w_\pi/p$
      \EndIf
      \If{$a_0 \le a$}
        \State $i_* \gets i$
        \State $a \gets a_0$
      \EndIf
    \EndFor
    \State Schedule tasks $\tau_1, \ldots, \tau_{i_*}$ in
    parallel, distributing their work equally among the
    processors.
    \State $j \gets 0$
    \For{$k \in \{i_* + 1, \ldots, n\} $}
    \State Schedule $\tau_{k}$ in serial on processor $\rho_{1+(j \bmod p)}$
    \EndFor
  \end{algorithmic}
\end{algorithm}

Put simply THRESH schedules the $i_*$ tasks with largest serial
work in parallel, distributing their work equally, and schedules
the rest of the tasks in serial, sequentially giving out the
tasks, for the optimal value of $i_*$.

Clearly THRESH requires space $\Theta(1)$ beyond the input to
implement, and has running time $\Theta(n)$, given that the input
is pre-sorted.

In order to remove the assumption that the input is pre-sorted,
we can sort the tasks at the start of the algorithm, using heap
sort or radix sort. If we simply use heap sort then we clearly
will have running time $\Theta(n \log n)$ but still have $O(1)$
space requirement, and have competitive ratio $2$ with FROST.
On the other hand, if we use radix sort, it is conceivable that
we could do better. First, if the tasks can be assumed to have
serial works that can only take on $2^{O(1)}$ possible values,
then the radix sort can be performed in linear time, in-place,
resulting in a $2$-competitive algorithm that uses $O(1)$ memory
and $O(n)$ time. On the other hand, if such an assumption cannot
be made, then if better running time is still desired, then it
can be achieved at a cost to the competitive-ratio. Let $1$ be
the largest serial work of any task. If we sort the tasks based
on the key of which bucket $[1/2^i, 1/2^{i-1}]$ the task's
serial work falls in, except saying
that any tasks with work less than $1/2^{\lg (np)}$ fall in the
same bucket, then there are only $\log n + \log p$ distinct keys,
so sorting can be done much faster. If we are willing to spend
$\Theta(n + \log (np))$ space, then the sorting can be done by
counting sort in time $\Theta(n+\log(np))$. On the other hand,
using radix sort we could do the sort in time $\Theta(n \log
(np))$ using only $\Theta(1)$ auxiliary space. 
Since the tasks are only approximately sorted we can only
guarantee $4$-competitiveness in this case: this is clear by
noting that increasing the serial works of all tasks by a factor
of $2$ would double the best attainable awake time.

Now we show why THRESH is $2$-competitive.
Consider OPT's strategy.
Let $\tau_{i_0}$ be the task with the largest serial work that
OPT schedules in serial. Recalling that the tasks are sorted by
serial work, for all $i < i_0$ we have that OPT chooses to
schedule $\tau_i$ in parallel. Let $w^{OPT}_\sigma = \sum_{i \ge
i_0} \sigma(\tau_i), w^{OPT}_\pi = \sum_{i < i_0} \pi(\tau_i)$.
OPT's awake time is obviously at least $(w^{OPT}_\pi +
w^{OPT}_\sigma)/p$. OPT's awake time is also obviously at least
$\sigma(\tau_{i_0})$.

Think about the sequential thing.
Ignore $\sigma(\tau_{i_*})$. Then $\rho_i$ always has less work
than all the other processors. Of course in reality $\rho_1$ has
the most work of any processor.
So we have that all processors have serial work at most 
$$\sum_{i > i_*} \sigma(\tau_i)/p + \sigma(\tau_{i_*}).$$
Add on to that $$\sum_{i \le i_*} \pi(\tau_i)/p.$$

If $x \le a + b$, and $y \ge a, y \ge b$, then obviously $x \le
2y$. Thus, THRESH is $2$-competitive with FROST. 


\section{Lower Bounds}
\label{sec:lowerbounds}

In this section we prove several impossibility results, which
show that we cannot hope to substantially improve our algorithms.

\subsection{Deterministic Algorithms for Minimizing Awake Time}
It is clear that GR is not $(2-\epsilon)$-competitive for any
$\epsilon > 0$. We might hope to achieve a
$(1+\epsilon)$-competitive scheduling algorithm for this problem.
However, in this subsection we establish that it is impossible
for an off-line deterministic scheduler to get a competitive
ratio lower than $1.25$, even using preemption. That is, we show
that for any deterministic algorithm ALG there is some input on
which ALG has awake time at least $1.25$ times greater than OPT. 

In Table~\ref{tab:lowerboundFork1} and
Table~\ref{tab:lowerboundFork2} we specify two TAPs.
For each time we give a list of which tasks arrive in the format
$(\sigma, \pi)\times m$ where $\sigma, \pi$ are the serial and
parallel works of a task and $m$ is how many of this type of task
arrive at this time.

\begin{table}[H]
\caption{}
\label{tab:lowerboundFork1}
\centering
\begin{tabular}{|l|l|}
\hline
time & tasks                    \\ \hline
$0$  & $(4, 2p) \times 1$       \\ \hline
$1$  & $(3, 3p) \times (p-1)$ \\ \hline
\end{tabular}
\end{table}

\begin{table}[H]
\caption{}
\label{tab:lowerboundFork2}
\centering
\begin{tabular}{|l|l|}
\hline
time & tasks                    \\ \hline
$0$  & $(4, 2p) \times 1$       \\ \hline
\end{tabular}
\end{table}

Consider an arbitrary deterministic scheduling algorithm. If at
time $0$ the arriving tasks are $(4, 2p)\times 1$ (i.e. a single
task arrives, with serial work $4$ and parallel work $2p$) then
the scheduler has two options: it can schedule this task in
serial, or in parallel.

If no further tasks arrive, i.e. the task schedule is from
Table~\ref{tab:lowerboundFork2} then OPT would have awake time
$2$ by distributing the tasks work equally amongst all
processors, whereas a scheduler that ran the task in serial for
all of the time that it was running the task during the first
second after the task arrived would have awake time at least $3$.
In this case the competitive ratio of the algorithm is at least $1.5$.

On the other hand, the algorithm could decide to not run the task
in serial for any time during the first second after the task
arrives. In this case, if
and it turns out that the task schedule is from
Table~\ref{tab:lowerboundFork1}, then the algorithm has again
acted sub-optimally. In particular, for the schedule given in
Table~\ref{tab:lowerboundFork1}, OPT schedules the task that
arrives at time $0$ in serial, and then schedules all the tasks
that arrive at time $1$ in serial as well, and hence achieves
awake time $4$. On the other hand, the awake time of an algorithm
that did not schedule the task that arrived at time $0$ in
serial is at least $5$: such a scheduler may either choose at
time $1$ to cancel the task from time $0$ and run it in serial,
or the scheduler may choose to let the parallel implementation
finish running. In this case the competitive ratio of the
algorithm is $5/4$.

Hence it is impossible for any deterministic algorithm to achieve
a competitive ratio of lower than $1.25$.

We remark that the numbers in this argument can clearly be
optimized, to give an improved lower bound of about $1.36$ on
competitive ratio. As this is asymptotically not interesting, and
much messier, we decide to not give this better argument.

\subsection{Randomized Algorithms For Minimizing Awake Time}
We might that there is a randomized algorithm that gets a
competitive ratio substantially better than any deterministic
algorithm can, for example maybe there is a randomized algorithm
that is $(1+\epsilon)$-competitive on any input with high
probability, or a randomized algorithm with expected competitive
ratio at most $(1+\epsilon)$ on any input. However, in this
subsection we show that this is impossible.

In particular, we demonstrate a lower-bound of $1.0625$ on the
competitive ratio of any randomized off-line algorithm.

Recall the TAPs from Table~\ref{tab:lowerboundFork1} and
Table~\ref{tab:lowerboundFork2}; we will use these as sub-parts
of our the TAP that we build to be adversarial for a randomized
algorithm. 

Fix some off-line randomized algorithm RAND. We say that an input
TAP is \defn{bad} for RAND if with high probability RAND's awake
time on TAP is at least $1.0625$ times that of OPT.
We construct a class of TAPs, and show that some of the TAPs in
this class must be bad for RAND.

Let $\mathcal{T}_{I}$, for some some binary string $I$, be the
TAP consisting of the TAP from Table~\ref{tab:lowerboundFork1} at
time $10i$ if $I_i = 1$ and the TAP from
Table~\ref{tab:lowerboundFork2} if $I_i = 0$. 

Consider a $I$ chosen uniformly at random from $\{0,1\}^m$ for
some parameter $m$.
On each sub-tap RAND has at most a $1/2$ chance of acting as OPT
does, and at least a $1/2$ chance of acting sub-optimally, in
particular, from our analysis above showing that any deterministic
algorithm has competitive ratio at least $1.25$ on at least one
of these inputs, RAND has at least a $1/2$ chance of this
happening.
By a Chernoff Bound, with probability at least
$1-e^{-\Omega(m)}$, on at least $1/4$ of the sub-taps RAND has
competitive ratio at least $1.25$. Since there is no overlap, by
design, of the sub-taps (by spacing them out), this means that
overall the competitive ratio of RAND is at least $1\cdot 3/4 +
1.25 \cdot 1/4 = 1.0625.$

Note that the number of tasks in such a TAP is less than $mp$, so
$n \le mp$, and thus $m \ge n/p$.
Hence our result that holds with high probability in $m$ holds
with high probability in $n/p$ too.
Of course $n\gg p$ so this is pretty decent.

Because a randomly chosen TAP from this class of TAPs is bad for
RAND with high probability in $n/p$, by the probabilistic method
there is at least one TAP in this class of TAPs that is bad for
RAND. 

\subsection{Preemption is necessary for Minimizing Mean Response Time}

In this paper we consider the metric of awake time. Another
possible metric is mean response time. In this subsection we
briefly demonstrate a major difference between the problem of
minimizing mean response time and minimizing awake time:
Preemption is necessary for Minimizing Mean Response Time.

Consider a deterministic scheduling algorithm ALG that does not
use preemption. Say that the $\max$ number of processors given
work over all input TAPs is $p_0$. We claim that there is some
input TAP on which ALG does arbitrarily poorly compared to OPT in
terms of mean response time.
Consider a sequence of tasks that forces ALG to have $p_0$
processors in use, and let $h_0$ be the minimum amount of work on
any processor with work. Say we have sent $n_0$ tasks so far.
We choose $n$ such that $n_0 = \epsilon n$, and now we send
$(1-\epsilon)n$ tasks each with work $h_0/2$. OPT is presumably
going to be preempting stuff to run these, so our competitive
ratio is basically $\Theta(n)$, which is in particular trash.

\subsection{Unknown dependencies}
In the unknown dependencies it is impossible to get an
$O(1)$-competitive algorithm with the optimal algorithm that
knows the dependencies.
In particular, we give a DTAP with $n=p$ tasks, that can be
processed in time $O(1)$ by an algorithm that knows the
dependencies, but that takes time
$\Omega(\sqrt{n})$ to be processed in the worst
case by any deterministic algorithm that does not know the
dependencies.
In fact, we can show any deterministic or randomized algorithm
takes time $\Omega(\sqrt{n})$ to process this DTAP with high
probability in $n$.

We now construct and analyze the DTAP. We will make $n=p$ tasks.
The DTAP consists of $\sqrt{p}$ \defn{levels}. On each level of
the DTAP the tasks are $(1,\sqrt{p})\times \sqrt{p}$. One of
these tasks spawns the $\sqrt{p}$ tasks at the next level of
recursion upon completion.

A scheduler that knew the dependencies would first run all the
spawning tasks using their parallel implementations. Doing so the
emptier could unlock all the tasks in time $\sqrt{p} \sqrt{p} / p
= 1$. Then there are less than $p$ tasks, so the scheduler can
schedule them all with their serial implementations, and finishes
$1$ unit of time later. In total this gives awake time $2$.
But if the scheduler does not know which are the spawning tasks,
then it can't immediately do them. In the worst case such a
scheduler will take $1$ unit of time to uncover the dependencies:
if the scheduler runs any task in serial it could take $1$ unit
of time, and if all tasks are run in parallel it could take
$\sqrt{p}\sqrt{p} / p = 1$ unit of time.
Taking $1$ unit of time on each of the $\sqrt{p}$ levels means
that this scheduler has awake time at least $\sqrt{p} = \sqrt{n}$.

By doing tasks in parallel until the spawning task is uncovered a
randomized emptier can hope to do slightly better. But it turns
out not better asymptotically.
With high probability in $n$ the scheduler takes time at least
$1/2$ on at least $1/4$ of the levels.
Hence the scheduler takes time at least $1/8 \sqrt{n}$ with high
probability in $n$, as desired.


\section{Recursion}
\label{sec:recursion}
\todo{
First we must formalize this problem. Like what does
this even mean?
}


\bibliographystyle{plain}
\bibliography{paper}

\end{document}

\documentclass[twocolumn]{article}[10pt]
\usepackage[left=1in, right=1in, top=1in, bottom=1in]{geometry}
% \usepackage[subtle]{savetrees} 

\usepackage{amsthm}
\usepackage{amssymb}
\usepackage{amsmath}
\usepackage{mathtools}
\usepackage{xcolor}
\usepackage{xspace}
\usepackage{csquotes} 

\usepackage{fancyhdr}
\pagestyle{fancy}
\fancyhead{}
\fancyfoot{}
\fancyfoot[R]{\thepage}
\renewcommand{\headrulewidth}{0pt}

\usepackage{hyperref}

\newcommand{\defn}[1]{{\textit{\textbf{\boldmath #1}}}\xspace}
\renewcommand{\paragraph}[1]{\vspace{0.09in}\noindent{\bf \boldmath #1.}} 
\newcommand{\todo}[1]{{\color{red}\textbf{TODO:} #1}}

\DeclareMathOperator{\E}{\mathbb{E}}
\DeclareMathOperator{\Var}{\text{Var}}
\DeclareMathOperator{\img}{Img}
\DeclareMathOperator{\polylog}{\text{polylog}}
\newcommand{\defeq}{\vcentcolon=}
\newcommand{\eqdef}{=\vcentcolon}

\DeclareMathOperator{\work}{\text{work}}

\newcommand{\floor}[1]{\left\lfloor #1 \right\rfloor}
\newcommand{\ceil}[1]{\left\lceil #1 \right\rceil}
\newcommand{\paren}[1]{\left( #1 \right)}

\usepackage{algorithm}
\usepackage[noend]{algpseudocode} % adding [noend] deletes the end while and stuff

\usepackage[capitalise,nameinlink,noabbrev]{cleveref}
\crefname{equation}{}{} % cref{eq:blah} only does (1) instead of Equation (1)
\crefname{enumi}{Step}{} % cref{eq:blah} only does (1) instead of Equation (1)

\newtheorem{fact}{Fact}
\newtheorem{definition}{Definition}
\newtheorem{remark}{Remark}
\newtheorem{proposition}{Proposition}
\newtheorem{claim}{Claim}
\newtheorem{conjecture}{Conjecture}
\newtheorem{lemma}{Lemma}
\newtheorem{corollary}{Corollary}
\newtheorem{theorem}{Theorem}

\author{Alek Westover}
\title{Serial-Parallel Scheduling Problem}

\begin{document}
\maketitle

\begin{abstract}
  There are many problems for which the best parallel algorithms
  have larger cost than the best serial algorithms. 
  We consider a scheduler that is receiving many tasks with
  serial and parallel implementations that have potentially
  different costs. The scheduler can choose whether to run each
  task in serial or in parallel.
  The scheduler aims to minimize the total time that it has
  unfinished tasks. We analyze the competitive ratio of
  schedulers, i.e. the ratio of the time of a scheduler to the
  optimal time.

  We construct a simple deterministic scheduler that does not use
  preemption that is $2$-competitive for the scheduling problem.
  We also prove that no deterministic scheduler can have a
  competitive ratio smaller than $1.44$ in general.
  We also exhibit a randomized scheduler that achieves
  expected competitive ratio at least $1.5$.

  Also, we look at the problem when the tasks are allowed to do
  recursion, i.e. they can spawn multiple tasks. In this case we
  prove XXX.

\end{abstract}

\section{Introduction}
\subsection{Problem Specification}
A parallel algorithm is said to be \defn{work-efficient} if the
work of the parallel algorithm is the same as the work of a
serial algorithm for the same problem. Most implementations of
parallel algorithms are not work-efficient, often having work
that is a constant factor greater, or even asymptotically
greater, than the work of the serial algorithm for the problem.

In the \defn{Serial-Parallel Scheduling Problem} we have to
perform $n$ tasks $\tau_1, \ldots, \tau_n$ ($n$ unknown ahead of
time). We have $p$ processors $\rho_1, \ldots, \rho_p$. Each task
$\tau_i$ has a parallel implementation with work $\pi(\tau_i)$ and a
serial implementation with work $\sigma(\tau_i)$. The tasks will become
available at some times $t(\tau_1), \ldots, t(\tau_n)$. 
The sequence of tasks with their associated parallel and serial
implementations works and with their associated arrival times is
called a \defn{task arrival plan} or \defn{TAP} for short.
Note that we consider time to be continuous, although our
algorithms and bounds actually also apply if tasks must arrive
and finish at discrete times; in fact having discrete time seems
to hurt OPT much more than an off-line algorithm.

The scheduler maintains a set of \defn{ready} tasks, which are
tasks that have become available but are not currently being run
on any processor. At time $t(\tau_i)$ task $\tau_i$ is added to
the set of ready tasks. At any time the scheduler can decide to
schedule some (not already running) ready task, and can choose
whether to run the task in serial, in which case the scheduler
must choose a single processor to run the task on, or in
parallel, in which case the scheduler can distribute the task's
work arbitrarily among the processors. Intuitively, if there are
many ready tasks then the scheduler should run the serial
implementations of the tasks because the scheduler can achieve
parallelism across the tasks. On the other hand, if there are not
very many ready tasks it is probably better for the scheduler to
run the parallel versions of the tasks --- even though they are
possibly not work efficient, i.e. $\pi(\tau) > \sigma(\tau)$ ---
because by so doing at least the scheduler can achieve
parallelism within tasks.

Let the \defn{awake time} of the scheduler be the duration of
time over which the scheduler has unfinished tasks.
The scheduler attempts to minimize awake time.
We measure how well the scheduler is able to minimize its awake
time by comparing its awake time to the awake time of the optimal
strategy, which we will denote OPT. Note that OPT is able to see
the whole sequence of tasks in advance.
The \defn{competitive ratio} of a scheduler is the ratio
of its awake time to the awake time of OPT on the same input.

\subsection{Problem Motivation}
Data-centers often get heterogeneous tasks. Being able to schedule
them efficiently is a fundamental problem. 

\subsection{Related Work}
An algorithm called Shortest Remaining Processing Time (SRPT),
and its variants are often useful for minimizing metrics like mean
response time.

In \cite{bb20}, Berg et al study a related problem:
many heterogeneous tasks come in, some which are elastic and
exhibit perfect linear scaling of performance and some which are
inelastic which must be run on a single processor, according to
some stochastic assumptions, and they aim to minimize mean
response time.

In \cite{is16} Im et al exhibit an algorithm keeps the average
flow time small. 

In \cite{ga12} Gupta et al prove some impossibility results about
a problem somewhat similar to our problem.

Clearly related problems are widely studied.
Our problem is novel however, and interesting.

\subsection{Results}
We construct a simple deterministic scheduler that does not use
preemption that is $2$-competitive for the scheduling problem.
We also prove that no deterministic scheduler can have a
competitive ratio smaller than $1.44$ in general.
We also exhibit a randomized scheduler that achieves
expected competitive ratio at least $1.5$.

Also, we look at the problem when the tasks are allowed to do
recursion, i.e. they can spawn multiple tasks. In this case we
prove XXX.

\section{A Deterministic Scheduling Algorithm}
In this section we present a simple deterministic scheduling
algorithm that does not use preemption. We show that this
algorithm is $2$-competitive with OPT.

First we note that without loss of generality we may consider
TAPs where the cost ratio $\pi(\tau)/\sigma(\tau)$ for any task
$\tau$ is in $[1,p]$; if $\pi(\tau)/\sigma(\tau) < 1$, i.e. the
parallel implementation has lower work than the serial
implementation, then the scheduler clearly should never use the
serial implementation of this algorithm, so we can replace the
serial implementation with the parallel implementation and hence
get cost ratio $1$, similarly, if $\pi(\tau)/\sigma(\tau) > p$
then the scheduler should never run the parallel task and we can
replace the parallel implementation of the task with the serial
implementation to get cost ratio $p$.

We say that a time is a \defn{verge} time for our algorithm if at
this time no processors are performing tasks and there is at
least one ready task.

We propose Algorithm~\ref{alg:gr}, which we call \defn{GR}
(GR is short for \enquote{greedy}), as a scheduling algorithm.

\begin{algorithm}
  \caption{GR}
  \label{alg:gr}
  \begin{algorithmic}
    \While{True}
      \If{verge time}
        \State do what OPT would do in the case where no tasks arrive later than
        this verge time in the TAP 
      \EndIf
    \EndWhile
  \end{algorithmic}
\end{algorithm}

GR is an incredibly simple algorithm, although there is some
hidden complexity in it: the algorithm is consulting an oracle to
know what OPT would do, albeit in the relatively simple special
case of where all the tasks arrive at the same time. In
Corollary~\ref{cor:one_round_OPT_oracle} we exhibit an oracle
that yields a schedule that achieves the same awake time as OPT
for TAPs where all the tasks arrive at a single time, hence
showing that GR actually can be computed. First we analyze the
competitive ratio of GR assuming the existence of such an oracle. 

Consider a TAP $\mathcal{T}$. Let $\ell$ be the number of verge
times for $\mathcal{T}$; note that $\ell \le n$ which in
particular is finite. Let $t_i$ be the $i$-th time that is a
verge time, let $q_i$ be the number of ready tasks for GR at
time $t_i$. Let $T^{ALG}(q_1, \ldots, q_{\ell'})$ denote the
awake time of a scheduling algorithm ALG on the truncation of the
TAP $\mathcal{T}$ that only consists of tasks arriving at times
before $t_{\ell'}$.

By construction we have
\begin{equation}
  \label{eq:same_single}
  T^{OPT}(q) = T^{GR}(q).
\end{equation}
We remark GR \enquote{locally} schedules optimally, which is why
we refer to GR as \enquote{greedy}. 

An ALG-gap is an interval of time of non-zero length where for
all times in the interior of the interval ALG has completed every
task that has arrived thus far. Additionally for an interval to
be an ALG-gap the interval must contain no other intervals which
are also ALG-gaps (i.e. it is a \enquote{maximal} interval
satisfying our conditions).
We say that a TAP is \defn{ALG-gap-free} if it contains no ALG-gaps.

Now we prove an important property of OPT.
\begin{claim}
  \label{clm:OPT_finishes_first}
  If there is a scheduling algorithm ALG that completes all tasks by
  time $t_*$ then OPT finishes all tasks by time $t_*$.
\end{claim}
\begin{proof}
  Say that ALG completes all tasks by time $t_*$. Let $t_0 < t_*$
  be the most recent time that OPT has completed all tasks that
  arrive before time $t_0$. If OPT has not finished all tasks by
  time $t_*$ then it was acting sub-optimally, as it could steal
  the strategy that ALG used on $[t_0, t_*]$ to achieve lower
  awake time. In particular, for any tasks that arrive in $[t_0,
  t_*]$ OPT could schedule them as ALG schedules them. We remark
  that OPT should not steal all of ALG. 
\end{proof}
Note that as an immediate consequence of Claim~\ref{clm:OPT_finishes_first}
we have that any ALG-gap is a subset of an OPT-gap.

Decomposing TAPs into gap-free subsets of the TAP is very useful.
Part of the reason for this is the following fact:
\begin{claim}
  \label{clm:just_consider_gapless}
  If an algorithm ALG achieves competitive ratio $r$ on
  ALG-gap-free TAPs, then ALG achieves 
  competitive ratio $r$ on arbitrary TAPs.
\end{claim}
\begin{proof}
  We partition the tasks based on arrival time, splitting the
  tasks on the ALG-gaps. That is, we split the tasks into groups
  so that two tasks $\tau_i, \tau_j$ are in the same group if and
  only if there are no gaps in between the arrival times of
  $\tau_i$ and $\tau_j$.
  We can define an interval of time $I_i$ for each of these
  ALG-gap-free subsets of the TAP, where $I_i$ is defined so that
  all tasks in the $i$-th group start and finish at times
  contained in the interval $I_i$.

  Let $T_{I_i}^{OPT}$ and $T_{I_i}^{ALG}$ denote the awake time
  of OPT and ALG on interval $I_i$. Because $I_i$ is ALG-gap-free
  we have $T^{ALG} = \sum_{i} T^{ALG}_{I_i}$.
  Further, recall that by Claim~\ref{clm:OPT_finishes_first} any
  ALG-gap is also an OPT-gap, so
  $T^{OPT} = \sum_{i} T_{I_i}^{OPT}$. 
  Hence from our assumption that ALG is $r$-competitive on
  gap-free TAPs, such as the subset of the TAP on the interval
  $I_i$, we have $T_{I_i}^{ALG} \le r\cdot T_{I_i}^{OPT}$ for
  all $i$. Summing we get $T^{ALG} \le r\cdot T^{OPT}$, as desired.
  
\end{proof}

By Claim~\ref{clm:just_consider_gapless}, in order to bound GR's
competitive ratio, it suffices to consider TAPs
without GR-gaps. Note however that a TAP without
GR-gaps could still have OPT-gaps.

We conclude our analysis of the competitive ratio of GR in
Proposition~\ref{prop:2competitive} with an inductive argument on
the number of OPT-gaps in the TAP.
First we establish the base case for the argument: we consider
GR's competitive ratio on a TAP without OPT-gaps.  

\begin{claim}
  \label{clm:no_optgaps}
  GR is $2$-competitive on any OPT-gap-free TAP.
\end{claim}
\begin{proof}
  Since the TAP is OPT-gap-free we must have
  \begin{equation}
    \label{eq:opt_isnt_so_much_better}
    T^{OPT}(q_1, \ldots, q_{\ell}) \ge T^{SGR}(q_1, \ldots, q_{\ell-1}).
  \end{equation}
  Because GR finishes all $q_{i}$ tasks that arrive at time $t_i$
  by time $t_{i+1}$ we can actually always decompose
  $T^{GR}(q_1, \ldots, q_\ell)$ as 
  \begin{equation}
    \label{eq:decomposeGR}
    T^{GR}(q_1, \ldots, q_\ell) = \sum_{i=1}^\ell T^{GR}(q_i).
  \end{equation}
  By Equation~\eqref{eq:decomposeGR}, and
  Equation~\eqref{eq:same_single} we thus have 
  \begin{equation}
    \label{eq:decompose_rearanged}
    T^{GR}(q_1, \ldots, q_\ell) = T^{GR}(q_1, \ldots, q_{\ell-1}) + T^{OPT}(q_\ell).
  \end{equation}

  Hence by Equation~\eqref{eq:opt_isnt_so_much_better} and
  Equation~\eqref{eq:decompose_rearanged} we have
  \begin{align*}
    T^{GR}(q_1, \ldots, q_\ell) &\le T^{OPT}(q_1, \ldots, q_\ell) + T^{OPT}(q_\ell)\\
                                   &\le 2T^{OPT}(q_1, \ldots, q_\ell),
  \end{align*}
  as desired.
\end{proof}

\begin{proposition}
  \label{prop:2competitive}
  GR is $2$-competitive.
\end{proposition}
\begin{proof}
  The proof is by strong induction on the number of OPT-gaps. 
  The base case of our induction is established in
  Claim~\ref{clm:no_optgaps}, which says that if there are $0$
  OPT-gaps then GR is $2$-competitive. 

  Consider a TAP that has more than $0$ OPT gaps;
  say that its first OPT-gap starts at time $t_*$.
  Let $j$ be the largest index such that verge time $t_j <
  t_*$.

  Using our inductive hypothesis we have:
  \begin{align*}
  &T^{OPT}(q_1, \ldots, q_\ell) \\
  &\ge T^{OPT}(q_1, \ldots, q_j) + T^{OPT}(q_{j+1}, \ldots, q_{\ell})\\
  &\ge \frac{1}{2}\paren{T^{GR}(q_1, \ldots, q_j) + T^{GR}(q_{j+1}, \ldots, q_{\ell})}\\
  &=\frac{1}{2} T^{GR}(q_1, \ldots, q_\ell).
  \end{align*}

\end{proof}

Now we demonstrate the existence of an oracle computing OPT on
TAPs where all tasks arrive at a single point in time.
First, we need a lemma:
\begin{lemma}[Cake frosting lemma]
  you can frost a cake
\end{lemma}
\begin{proof}
  Of course by Claim~\ref{clm:sgr_single} $T^{OPT}(q) \le
  T^{SGR}(q) = T(q)$. We now claim that OPT can do no better than
  this. This is obvious if $q \le p$: if OPT schedules any task
  in serial then OPT has awake time at least $1$, and if OPT
  schedules everything in parallel then OPT has awake time at
  least $qk/p$; for $q\le p$ we have $T(q) = \min(1, qk/p)$, so
  OPT can do no better than SGR here. It is intuitively obvious
  that if there are $q> p$ tasks, then scheduling about
  $\floor{q/p}$ tasks in serial to each processor like SGR does
  is the right strategy; proving this is somewhat intricate
  however.

  Say OPT schedules $x$ of the $q$ tasks in serial, and
  schedules the remaining $q-x$ in parallel.
  Say that an optimal assignment of the tasks, under the constraint
  that $x$ of the tasks are scheduled in serial and $q-x$ are
  scheduled in parallel, achieves awake time $M$.

  We claim that there must exist an optimal assignment of tasks
  such that each processor is assigned either $\floor{x/p}$ or
  $\ceil{x/p}$ serial tasks. To prove this, we start from some
  optimal assignment, and modify it in such a way as to make the
  configuration \enquote{closer} to our desired configuration.
  Let $n_\sigma(\rho_i)$ be the number of serial tasks scheduled
  on processor $\rho_i$.

  To formalize a notion of \enquote{closeness} we define a potential 
  function $\phi$ of the assignment $S$ of tasks:
  $$\phi(S) = \sum_{i} \min(|n_\sigma(\rho_i)-\floor{x/p}|, |n_\sigma(\rho_i)-\ceil{x/p}|.$$ 
  Note that $\phi(S)$ is non-negative. We desire a configuration with $\phi(S) = 0$.
  Consider a configuration of tasks achieving awake time $M$ with $\phi(S) > 0$.
  We first apply the following procedure to the configuration:
  while the difference between the maximum work assigned to a
  processor and the minimum work assigned to a processor is more
  than $1$ swap $1$ unit of work from a processor with the
  maximum amount of work to a processor with the minimum amount
  of work. This swap cannot increase the range of works assigned
  to processors, and the process must eventually terminate.
  Now, while $\phi(S) > 0$ there must 
  exists $\rho_i, \rho_j$ with $n_\sigma(\rho_i) < \floor{x/p}$
  and $n_\sigma(\rho_j) > \floor{x/p}$ or there exists $\rho_i,
  \rho_j$ with $n_\sigma(\rho_i) < \ceil{x/p}$ and
  $n_\sigma(\rho_j) > \ceil{x/p}$. By construction the range of
  the works is at most $1$; hence $\rho_i$ has at least $1$ unit
  of parallel work to have work within $1$ of $\rho_j$ despite
  having at least $2$ fewer serial tasks than $\rho_j$. Then
  $\rho_i$ gives this $1$ unit of parallel work to $\rho_j$, and
  in exchange $\rho_j$ gives $\rho_i$ a serial task. This
  swapping operation decreases $\phi(S)$ by exactly $1$, and does
  not change the amount of work assigned to each processor, which
  importantly means that the swap does not increase the range of
  the works. Hence, we can repeat this swapping process to
  eventually achieve a configuration with awake time $M$ where each 
  processor has $\floor{x/p}$ or $\ceil{x/p}$ serial tasks.

  If $x\bmod p = 0$ then the awake time of OPT is clearly at
  least 
  \begin{equation} \label{eq:case_xmodp0}
    x/p + (q-x)k/p.
  \end{equation}
  Note that if $x$ increases by $p$ then \eqref{eq:case_xmodp0}
  changes by $1-k < 0$. That is, \eqref{eq:case_xmodp0} is
  monotonically decreasing in $x$, and is thus minimized by
  setting $x = p\floor{q/p}$ and hence getting $q-x = q\bmod p$.
  This gives awake time $\floor{q/p} + (q\bmod p)k/p \ge T(q)$.

  Now we consider the case where $x\bmod p \neq 0$. Here $(p -
  (x\bmod p))/k$ tasks can be added in serial without
  increasing the work at all. Thus the awake time is at least
  \begin{equation} \label{eq:case_xmodpnot0}
    \ceil{x/p} + \frac{k}{p}\paren{\max\paren{0, q-x - \frac{p-(x\bmod p)}{k}}}.
  \end{equation}

  Consider when the $\max$ in \eqref{eq:case_xmodpnot0} yields $0$. 
  For this to happen we must have
  $$q-x \le \frac{p-x\bmod p}{k}.$$
  As $p-x\bmod p \le p$ we get the following bound on $q-x$:
  $$q-x \le p/k,$$
  which must be met in order for the $\max$ to yield $0$.
  Clearly as $q \ge x \ge q-p/k$, $\floor{x/p} \ge \floor{q/p}-1$; we
  claim that this inequality is actually strict. Imagine that
  $\floor{x/p} = \floor{q/p}-1$. Because the $\max$ expression
  yields $0$, meaning that the parallel tasks add nothing to the
  awake time, if all processors only have $\floor{q/p}-1$ or
  $\floor{q/p} < q/p$ serial tasks, then the total work is at
  most $\floor{q/p} < q/p$ which is impossible: the total work in
  the system must be at least $q$ and it can be distributed in
  the best case perfectly equally which makes $q/p$ as a lower
  bound on the time to complete $q$ tasks. Hence $\floor{x/p} = \floor{q/p}$. 
  But then the awake time is at least $\ceil{x/p} = \floor{q/p} + 1 \ge T(q)$.

  Now we consider the case when the $\max$ in
  \eqref{eq:case_xmodpnot0} yields some positive number.
  Note that if $x$ increases by $p$ (but $x$ is still
  sufficiently small so that the $\max$ yields a positive number)
  then \eqref{eq:case_xmodpnot0} changes by $1-k < 0$.
  Further, if $x$ increases by $1$ (but $x$ is still
  sufficiently small so that the $\max$ yields a positive number) without $\ceil{x/p}$
  changing, then \eqref{eq:case_xmodpnot0} changes by $(k/p)(1/k
  -1) < 0$. That is, \eqref{eq:case_xmodpnot0} is monotonically
  decreasing in $x$. Hence we still have that the awake time is at least $T(q).$

  We have considered all cases, and shown that no matter what
  choice of $x$ OPT makes, and no matter how OPT schedules given
  that choice of $x$, OPT must incur awake time at least $T(q)$,
  as desired.
\end{proof}


\begin{algorithm}
  \caption{OPT Oracle}
  \label{alg:opt_oracle}
  \begin{algorithmic}
    \State $\text{minAwakeTime} \gets \infty$
    \For{$I \in \{0,1\}^n$} 
      \If{$I_i = 1$}
        \State Schedule task $i$ in parallel when it arrives
      \Else
        \State Schedule task $i$ in serial when it arrives
      \EndIf
      \State In particular, first schedule serial tasks to minimize awake time
      \State And then sprinkle parallel tasks on top 

      \If{$\text{awakeTime}_I \le \text{minAwakeTime}$}
        \State{$\text{minAwakeTime} \gets \text{awakeTime}_I$}
      \EndIf
    \EndFor
  \end{algorithmic}
\end{algorithm}

\begin{corollary}
  OPT Oracle is an oracle for OPT.
\end{corollary}
\begin{proof}
  
\end{proof}


\section{Lower Bounds}

In this section we prove several impossibility results, which
show that we cannot hope to substantially improve our algorithms.

\subsection{Deterministic Algorithms for Minimizing Awake Time}
It is clear that GR is not $(2-\epsilon)$-competitive for any
$\epsilon > 0$. We might hope to achieve a
$(1+\epsilon)$-competitive scheduling algorithm for this problem.
However, in this subsection we establish that it is impossible
for an off-line deterministic scheduler to get a competitive
ratio lower than $1.25$, even using preemption. That is, we show
that for any deterministic algorithm ALG there is some input on
which ALG has awake time at least $1.25$ times greater than OPT. 

In Table~\ref{tab:lowerboundFork1} and
Table~\ref{tab:lowerboundFork2} we specify two sets of tasks.
For each time we give a list of which tasks arrive in the format
$(\sigma, \pi)\times m$ where $\sigma, \pi$ are the serial and
parallel works of a task and $m$ is how many of this type of task
arrive at this time.

\begin{table}[H]
\caption{}
\label{tab:lowerboundFork1}
\centering
\begin{tabular}{|l|l|}
\hline
time & tasks                    \\ \hline
$0$  & $(4, 2p) \times 1$       \\ \hline
$1$  & $(3, 3p) \times (p-1)$ \\ \hline
\end{tabular}
\end{table}

\begin{table}[H]
\caption{}
\label{tab:lowerboundFork2}
\centering
\begin{tabular}{|l|l|}
\hline
time & tasks                    \\ \hline
$0$  & $(4, 2p) \times 1$       \\ \hline
\end{tabular}
\end{table}

Consider an arbitrary deterministic scheduling algorithm. If at
time $0$ the arriving tasks are $(4, 2p)\times 1$ (i.e. a single
task arrives, with serial work $4$ and parallel work $2p$) then
the scheduler has two options: it can schedule this task in
serial, or in parallel.

If no further tasks arrive, i.e. the task schedule is from
Table~\ref{tab:lowerboundFork2} then OPT would have awake time
$2$ by distributing the tasks work equally amongst all
processors, whereas a scheduler that ran the task in serial for
all of the time that it was running the task during the first
second after the task arrived would have awake time at least $3$.
In this case the competitive ratio of the algorithm is at least $1.5$.

On the other hand, the algorithm could decide to not run the task
in serial for any time during the first second after the task
arrives. In this case, if
and it turns out that the task schedule is from
Table~\ref{tab:lowerboundFork1}, then the algorithm has again
acted sub-optimally. In particular, for the schedule given in
Table~\ref{tab:lowerboundFork1}, OPT schedules the task that
arrives at time $0$ in serial, and then schedules all the tasks
that arrive at time $1$ in serial as well, and hence achieves
awake time $4$. On the other hand, the awake time of an algorithm
that did not schedule the task that arrived at time $0$ in
serial is at least $5$: such a scheduler may either choose at
time $1$ to cancel the task from time $0$ and run it in serial,
or the scheduler may choose to let the parallel implementation
finish running. In this case the competitive ratio of the
algorithm is $5/4$.

Hence it is impossible for any deterministic algorithm to achieve
a competitive ratio of lower than $1.25$.

We remark that the numbers in this argument can clearly be
optimized, to give an improved lower bound of about $1.36$ on
competitive ratio. As this is asymptotically not interesting, and
much messier, we decide to not give this better argument.

\subsection{Randomized Algorithms For Minimizing Awake Time}
We might that there is a randomized algorithm that gets a
competitive ratio substantially better than any deterministic
algorithm can, for example maybe there is a randomized algorithm
that is $(1+\epsilon)$-competitive on any input with high
probability, or a randomized algorithm with expected competitive
ratio at most $(1+\epsilon)$ on any input. However, in this
subsection we show that this is impossible.

In particular, we demonstrate a lower-bound of $1.0625$ on the
competitive ratio of any randomized off-line algorithm.

Recall the TAPs from Table~\ref{tab:lowerboundFork1} and
Table~\ref{tab:lowerboundFork2}; we will use these as sub-parts
of our the TAP that we build to be adversarial for a randomized
algorithm. 

Fix some off-line randomized algorithm RAND. We say that an input
TAP is \defn{bad} for RAND if with high probability RAND's awake
time on TAP is at least $1.0625$ times that of OPT.
We construct a class of TAPs, and show that some of the TAPs in
this class must be bad for RAND.

Let $\mathcal{T}_{I}$, for some some binary string $I$, be the
TAP consisting of the TAP from Table~\ref{tab:lowerboundFork1} at
time $10i$ if $I_i = 1$ and the TAP from
Table~\ref{tab:lowerboundFork2} if $I_i = 0$. 

Consider a $I$ chosen uniformly at random from $\{0,1\}^m$ for
some parameter $m$.
On each sub-tap RAND has at most a $1/2$ chance of acting as OPT
does, and at least a $1/2$ chance of acting sub-optimally, in
particular, from our analysis above showing that any deterministic
algorithm has competitive ratio at least $1.25$ on at least one
of these inputs, RAND has at least a $1/2$ chance of this
happening.
By a Chernoff Bound, with probability at least
$1-e^{-\Omega(m)}$, on at least $1/4$ of the sub-taps RAND has
competitive ratio at least $1.25$. Since there is no overlap, by
design, of the sub-taps (by spacing them out), this means that
overall the competitive ratio of RAND is at least $1\cdot 3/4 +
1.25 \cdot 1/4 = 1.0625.$

Note that the number of tasks in such a TAP is less than $mp$, so
$n \le mp$, and thus $m \ge n/p$.
Hence our result that holds with high probability in $m$ holds
with high probability in $n/p$ too.
Of course $n\gg p$ so this is pretty decent.

Because a randomly chosen TAP from this class of TAPs is bad for
RAND with high probability in $n/p$, by the probabilistic method
there is at least one TAP in this class of TAPs that is bad for
RAND. 

\subsection{Preemption is necessary for Minimizing Mean Response Time}

Consider a deterministic scheduling algorithm ALG that does not
use preemption. Say that the $\max$ number of processors given
work over all input TAPs is $p_0$. We claim that there is some
input TAP on which ALG does arbitrarily poorly compared to OPT in
terms of mean response time.
Consider a sequence of tasks that forces ALG to have $p_0$
processors in use, and let $h_0$ be the minimum amount of work on
any processor with work. Say we have sent $n_0$ tasks so far.
We choose $n$ such that $n_0 = \epsilon n$, and now we send
$(1-\epsilon)n$ tasks each with work $h_0/2$. OPT is presumably
going to be preempting stuff to run these, so our competitive
ratio is basically $\Theta(n)$, which is in particular trash.



% \section{Randomized Scheduling Algorithms}
Given a particular deterministic scheduling algorithm there will
be some inputs on which the algorithm will perform poorly. 
By employing randomization these worst case inputs can be
mitigated somewhat, at least in expectation.

We propose Algorithm~\ref{alg:rgr}, which we call
\defn{RGR} (RGR stands for \enquote{randomized greedy}), for this case.

\todo{this alg is outdated}
\begin{algorithm}
  \caption{RGR}
  \label{alg:rgr}
  \begin{algorithmic}
    \While{True}
      \If{verge time}
        \State sleep for a random amount of time, chosen
        uniformly at random from something, not really sure what 
        \State $q \gets $ number of ready tasks
        \If{$q \ge p/k$}
          \State schedule $\min(q, p)$ tasks in serial
          \State giving each processor at most $1$ task
        \Else
          \State schedule one task in parallel
          \State distributing its work equally 
        \EndIf
      \EndIf
    \EndWhile
  \end{algorithmic}
\end{algorithm}

\begin{proposition}
  The expectation of RGR's competitive ratio on any input
  is at least $1.5$.
\end{proposition}
\begin{proof}
\todo{
 hmmm. 
 }
\end{proof}

% \section{Recursion}
\todo{
First we must formalize this problem. Like what does
this even mean?
}


% \section{Conclusions}
% \todo{
% Greed is good. 
% An interesting question is: the whole not doing preemption thing
% seems pretty dumb, what's going on?
% }

\bibliographystyle{plain}
\bibliography{paper}

\end{document}

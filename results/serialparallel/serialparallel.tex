\documentclass[twocolumn]{article}[10pt]
\usepackage[left=1in, right=1in, top=1in, bottom=1in]{geometry}
% \usepackage[subtle]{savetrees} 

\usepackage{amsthm}
\usepackage{amssymb}
\usepackage{amsmath}
\usepackage{mathtools}
\usepackage{xcolor}
\usepackage{xspace}
\usepackage{csquotes} 

\usepackage{fancyhdr}
\pagestyle{fancy}
\fancyhead{}
\fancyfoot{}
\fancyfoot[R]{\thepage}
\renewcommand{\headrulewidth}{0pt}

\usepackage{hyperref}

\newcommand{\defn}[1]{{\textit{\textbf{\boldmath #1}}}\xspace}
\renewcommand{\paragraph}[1]{\vspace{0.09in}\noindent{\bf \boldmath #1.}} 

\DeclareMathOperator{\E}{\mathbb{E}}
\DeclareMathOperator{\Var}{\text{Var}}
\DeclareMathOperator{\img}{Img}
\DeclareMathOperator{\polylog}{\text{polylog}}
\newcommand{\defeq}{\vcentcolon=}
\newcommand{\eqdef}{=\vcentcolon}

\newcommand{\floor}[1]{\left\lfloor #1 \right\rfloor}
\newcommand{\ceil}[1]{\left\lceil #1 \right\rceil}
\newcommand{\paren}[1]{\left( #1 \right)}

\usepackage{algorithm}
\usepackage[noend]{algpseudocode} % adding [noend] deletes the end while and stuff

\usepackage[capitalise,nameinlink,noabbrev]{cleveref}
\crefname{equation}{}{} % cref{eq:blah} only does (1) instead of Equation (1)
\crefname{enumi}{Step}{} % cref{eq:blah} only does (1) instead of Equation (1)

\newtheorem{fact}{Fact}
\newtheorem{definition}{Definition}
\newtheorem{remark}{Remark}
\newtheorem{proposition}{Proposition}
\newtheorem{clm}{Claim}
\newtheorem{claim}{Claim}
\newtheorem{lemma}{Lemma}
\newtheorem{corollary}{Corollary}
\newtheorem{theorem}{Theorem}

\author{Alek Westover}
\title{Serial-Parallel Scheduling Problem}

\begin{document}
\maketitle

\begin{abstract}
  There are many problems for which the best parallel algorithms
  have larger cost than the best serial algorithms. 
  We consider a scheduler that is receiving many tasks with
  serial and parallel implementations that have potentially
  different costs. The scheduler can choose whether to run each
  task in serial or in parallel.
  The scheduler aims to minimize the total time that it has
  unfinished tasks. We analyze the competitive ratio of
  schedulers, i.e. the ratio of the time of a scheduler to the
  optimal time.

  We exhibit a scheduler that is $2$-competitive for the
  symmetric-task case of this problem, a scheduler that is
  $4$-competitive for the symmetric-cost-ratio case of this
  problem, and an algorithm that is $8$-competitive for the
  general case of this problem.
  We prove that no deterministic scheduler can have a competitive
  ratio smaller than $2$.
  We also exhibit a randomized scheduler that achieves
  expected competitive ratio at least $1.5$.

  Also, we look at the problem when the tasks are allowed to do
  recursion, i.e. they can spawn multiple tasks. In this case we
  prove XXX.

  Another interesting question is to build an oracle to compute
  OPT. 

\end{abstract}

\section{Introduction}
A parallel algorithm is said to be \defn{work-efficient} if the
work of the parallel algorithm is the same as the work of a
serial algorithm for the same problem. Most implementations of
parallel algorithms are not work-efficient, often having work
that is a constant factor greater, or even asymptotically
greater, than the work of the serial algorithm for the problem.

In the \defn{Serial-Parallel Scheduling Problem} we have to
perform $n$ tasks $\tau_1, \ldots, \tau_n$ ($n$ unknown ahead of
time). We have $p$ processors $\rho_1, \ldots, \rho_p$. Each task
$\tau_i$ has a parallel implementation with work $\pi(\tau_i)$ and a
serial implementation with work $\sigma(\tau_i)$. The tasks will become
available at some times $t(\tau_1), \ldots, t(\tau_n)$. 
The sequence of tasks with their associated parallel and serial
implementations works and with their associated arrival times is
called a \defn{task arrival plan}.
Note that we consider time to be continuous, although our
algorithms and bounds actually also apply if tasks must arrive
and finish at discrete times; in fact having discrete time seems
to hurt OPT much more than an off-line algorithm.
The scheduler maintains a set of \defn{ready} tasks, which are tasks that have
become available but are not currently being run on any
processor. At time $t(\tau_i)$ task $\tau_i$ is added to the set of
ready tasks. At any time the scheduler can decide to schedule
some ready task, and can choose whether to run the task in
serial, in which case the scheduler must choose a single
processor to run the task on, or the scheduler can choose to run
the task in parallel, in which case the scheduler can distribute
the tasks work arbitrarily among the processors. Intuitively, if
there are many ready tasks then the scheduler should run the
serial implementations of the tasks because the scheduler can
achieve parallelism across the tasks. On the other hand, if there
are not very many ready tasks it is probably better for the
scheduler to run the parallel versions of the tasks --- even
though they are possibly not work efficient, i.e. $\pi(\tau) >
\sigma(\tau)$ --- because by so doing at least the scheduler can
achieve parallelism within tasks.

Let the \defn{awake time} of the scheduler be the duration of
time over which the scheduler has unfinished tasks.
The scheduler attempts to minimize awake time.

We measure how well the scheduler is able to minimize its awake
time by comparing its awake time to the awake time of the optimal
strategy, which we will denote OPT. Note that OPT is able to see
the whole sequence of tasks in advance.
The \defn{competitive ratio} of a scheduler is the ratio
of its awake time to the awake time of OPT on the same input.

\section{Discrete Time}

\section{Deterministic Scheduling Algorithms}

In this section we exhibit three scheduling algorithms that
guarantee small competitive ratios. We start with looking at
special cases of the game, and build on the strategies from the
special cases to get algorithms that work in more general
settings.

\subsection{Symmetric-Tasks Case}
\label{subsec:symmetrictasks}
First we consider a special case of the problem: the case where
all tasks have identical serial and parallel works. Let
the work of the serial implementations be $1$, and let the work
of the parallel implementations be $k \in [1, p]$\footnote{Note that
without loss of generality $k \in [1,p]$: if $k < 1$, i.e. the
parallel implementation has lower work than the serial
implementation, then the scheduler clearly should never use the serial
implementation of this algorithm, so we can replace the serial
implementation with the parallel implementation and hence get
$k=1$, similarly, if $k > p$ then the scheduler should never run
the parallel task and we can replace the parallel implementation
of the task with the serial implementation.}. Throughout this
subsection it is implicit that any task arrival plan we
refer to consists of identical tasks with serial work $1$ and parallel work $k$.

We say that a time is a \defn{verge} time for our algorithm if at
this time no processors are performing tasks and there is at
least one ready task.

We propose Algorithm~\ref{alg:sgr}, which we call \defn{SGR}
(SGR stands for \enquote{simple greedy}),
for scheduling in the symmetric-tasks case.

\begin{algorithm}
  \caption{SGR}
  \label{alg:sgr}
  \begin{algorithmic}
    \While{True}
      \If{verge time}
        \State $q \gets $ number of ready tasks
        \If{$q \ge p/k$}
          \State schedule $\min(q, p)$ tasks in serial
          \State giving each processor at most $1$ task
        \Else
          \State schedule $q$ tasks in parallel
          \State distributing work equally 
        \EndIf
      \EndIf
    \EndWhile
  \end{algorithmic}
\end{algorithm}

% DO NOT DELETE
% {\color{red} Similar alternative algorithms:
%   \begin{itemize}
%     \item SGR0: if $q < p/k$ schedule $1$ task in parallel.
%     \item SGR0 is nice because its clearly at least as good as SGR.
%     \item SGR2: schedule $\floor{q/p}$ tasks to each processor in
%   serial, then $q' = q\bmod p$, schedule $q'$ tasks in serial or
%   parallel (whichever minimizes awake time).
%     \item SGR2 is kind of nice because it means that for SGR2 there
%   is never any task overlap. I don't like task overlap. We'll see
%   how SGR tries to WLOG away task overlap, I don't think it's
%   going to work.
%   \end{itemize}
% }

In the rest of this subsection we analyze the competitive ratio
of SGR.

\begin{clm}
  \label{clm:OPT_finishes_before_you}
  If there is a scheduling algorithm ALG that completes all tasks by
  time $t_*$ then OPT finishes all tasks by time $t_*$.
\end{clm}
\begin{proof}
  Say that ALG completes all tasks by time $t_*$.
  Let $t_0 < t_*$ be the most recent time that OPT had no work.
  If OPT has work at time $t_*$ then it was acting sub-optimally,
  as it could steal the strategy that ALG used on $[t_0, t_*]$ to
  achieve lower awake time. In particular, for any tasks that
  arrive in $[t_0, t_*]$ OPT could schedule them as ALG schedules
  them. We remark that OPT cannot steal all of ALG. 
\end{proof}

Let $v$ be the number of verge times; note that $v\le n$ which
in particular is finite. Let $t_i$ be the $i$-th time that is a
verge time, let $q_i$ be the number of ready tasks at time
$t_i$, let $\Delta_i t = t_{i+1} - t_i$, let $\Delta_i q =
q_{i+1} - q_i$. Define $t_0 = -\infty, t_{v+1} = +\infty$; 
these are not verge times, but are merely defined for convenience.
We call the interval $(t_i, t_{i+1})$ for $i\in \{0,1,\ldots,
v\}$ \defn{valley interval $i$}. 

Let $T^{OPT}(q_1, \ldots, q_{v'})$ and $T^{SGR}(q_1, \ldots,
q_{v'})$ denote the awake time of OPT and SGR respectively on
the truncation of the task arrival plan that only consists of
tasks arriving at times before verge time $t_{v'}$ where the
task arrival plan is such that SGR has $q_i$ ready tasks at verge
time $t_i$ for all $i \le v'$. Let 
$$ T(q) = \floor{q/p} + \min(1, (q\bmod p)\cdot k/p). $$
We claim that when all work arrives before the first verge time
$T^{OPT}$ and $T^{SGR}$ are the same, and in particular are $T$.
\begin{clm}
  $$T(q) = T^{OPT}(q) = T^{SGR}(q).$$
\end{clm}
\begin{proof}
  If work only arrives before a single verge time, then all the
  work arrives at the same time. 

  SGR first schedules $p$ tasks in serial for $\floor{q/p}$
  verge times, which takes time $1\cdot \floor{q/p}$. Then on
  the final verge time SGR either schedules $q\bmod p$ tasks
  in serial, which takes time $1$, or schedules $q\bmod p$
  tasks in parallel, which takes time $(q\bmod p)k/p$; in
  particular SGR chooses whichever of these options leads to
  lower awake time. Hence overall on this task arrival plan
  SGR achieves awake time $T(q)$.

  We now claim that OPT can do no better than this. \textbf{This
  is pretty clear.} We remark that in this sense SGR is greedy: it
  \enquote{locally} schedules optimally. 

\end{proof}

An ALG-gap is a interval of time of non-zero length on which
the scheduling algorithm ALG has completed all tasks, but more
tasks have yet to arrive.
\begin{clm}
  \label{clm:just_consider_gapless}
  If an algorithm ALG can achieve competitive ratio $r$ on any
  task arrival plan with no ALG-gaps then it achieves
  competitive ratio $r$ on arbitrary task arrival plans.
\end{clm}
\begin{proof}
  If a task arrival plan has ALG-gaps, then we can partition the
  tasks based on arrival time, splitting the tasks into the same
  group if the interval of time book-ended by their arrival times
  has no ALG-gaps in it. In particular, we can define intervals
  $I_1, \ldots, I_g$ of time such that there are no ALG-gaps on
  $I_i$ for all $i\le g$, and such that some tasks arrive in $I_i$.
  By Claim~\ref{clm:OPT_finishes_before_you} an ALG-gap is also an
  OPT-gap. 
  Let $T_{I_i}^{OPT}$ and $T_{I_i}^{ALG}$ denote the awake time
  of OPT and ALG on interval $I_i$. Note that clearly
  $T^{OPT} = \sum_{i} T_{I_i}^{OPT}$ and $T^{ALG} = \sum_{i}
  T_{I_i}^{ALG}$. Hence if $T_{I_i}^{ALG} \le 2T_{I_i}^{OPT}$ for
  all $i$ then $T^{ALG} \le 2T^{OPT}$, as desired.
  
\end{proof}

By Claim~\ref{clm:just_consider_gapless}, in order to bound SGR's
competitive ratio, it suffices to consider task arrival plans
without SGR-gaps. Note however that a task arrival plan without
SGR-gaps could still have OPT-gaps.

\begin{clm}
  \label{clm:no_optgaps}
  SGR is $2$-competitive on any task arrival plan with no
  OPT-gaps.
\end{clm}
\begin{proof}
  Because \textbf{see my fairly convincing picture} we have
  \begin{align*}
    T^{OPT}(q_1, \ldots, q_{v}) &\ge T^{SGR}(q_2, \ldots, q_v) \\
  &\ge T^{SGR} - T(q_1)
  \end{align*}
  Hence
  \begin{align*}
    T^{SGR}(q_1, \ldots, q_v) &\le T^{OPT}(q_1, \ldots, q_v) + T(q_1) \\
                                   &\le 2T^{OPT}(q_1, \ldots, q_v).
  \end{align*}
\end{proof}

\begin{proposition}
  \label{prop:2competitive}
  SGR is $2$-competitive.
\end{proposition}
\begin{proof}
  The proof is by strong induction on the number of OPT-gaps. 
  The base case of our induction is established in
  Claim~\ref{clm:no_optgaps}, which says that if there are $0$
  OPT-gaps then SGR is $2$-competitive. 

  Consider a task arrival plan that has more than $0$ OPT gaps;
  say that its first OPT-gap starts at time $t_*$.
  Let $j$ be the largest index such that verge time $t_j <
  t_*$.

  Using our inductive hypothesis we have:
  \begin{align*}
  &T^{OPT}(q_1, \ldots, q_v) \\
  &\ge T^{OPT}(q_1, \ldots, q_j) + T^{OPT}(q_{j+1}, \ldots, q_{v})\\
  &\ge \frac{1}{2}\paren{T^{SGR}(q_1, \ldots, q_j) + T^{SGR}(q_{j+1}, \ldots, q_{v})}\\
  &=\frac{1}{2} T^{SGR}(q_1, \ldots, q_v).
  \end{align*}

\end{proof}


\subsection{Symmetric-Cost-Ratios Case}
\label{subsec:symmetriccostratio}
Next we consider the case where there are different tasks with
implementations that have different works, but with the
restriction that the cost ratio of the parallel implementation to
the serial implementation is some fixed value $k$.

The ideas in this section were inspired by extremely elegant
analysis of an unrelated scheduling problem in \cite{bamboo20}.

Making a global definition of $1$ unit of work is now difficult
to do in a meaningful way, so we do not do this. Instead, at
every verge time we define locally $1$ unit of work to be the
work of the serial implementation of the task with the serial
implementation with the most work. 
Further, we partition the unscheduled ready tasks at a given verge time into
sets called \defn{level-$i$} sets based on the work of their serial
implementation: the level-$i$ set of tasks on a verge time is the
unscheduled ready tasks that have serial implementation's with
work in $[1/2^{i+1}, 1/2^{i}]$.
We now define a \defn{virtual-task} to be a collection of tasks.
The work of the serial and parallel implementations of a
virtual-task are the sums of the works of the serial and parallel
implementations of the virtual-tasks constituent tasks.

We propose Algorithm~\ref{alg:lgr}, which we call \defn{LGR}
(LGR stands for \enquote{leveled greedy}),
for scheduling in the symmetric-cost-ratio case.

\begin{algorithm}
  \caption{LGR}
  \label{alg:lgr}
  \begin{algorithmic}
    \While{True}
      \If{verge time}
        \State Combine unscheduled ready tasks into virtual-tasks
        to maximize the number of level-$0$ virtual-tasks
        \State $q \gets $ number of virtual-tasks 
        \If{$q \ge p/k$}
          \State schedule $\min(q, p)$ virtual-tasks in serial
          \State giving each processor at most $1$ virtual-task
        \Else
          \State schedule a level-$0$ task in parallel
          \State distributing its work equally 
        \EndIf
      \EndIf
    \EndWhile
  \end{algorithmic}
\end{algorithm}

We prove the following regarding LGR:

\begin{proposition}
  LGR is $4$-competitive.
\end{proposition}


\subsection{General Case}
Now we are ready to consider the general case, i.e. we place no
restrictions on the tasks in this Subsection.
We use the definitions from Subsection
\ref{subsec:symmetriccostratio} and Subsection
\ref{subsec:symmetrictasks}.
We propose Algorithm~\ref{alg:ggr}, which we call
\defn{GGR} (GGR stands for \enquote{general greed}).

\begin{algorithm}
  \caption{GGR}
  \label{alg:ggr}
  \begin{algorithmic}
    \While{True}
      \If{verge time}
        \State Combine unscheduled ready tasks into virtual-tasks
        to maximize the number of level-$0$ virtual-tasks
        \State $q \gets $ number of virtual-tasks 
        \If{$q \ge p/k$}
          \State schedule $\min(q, p)$ virtual-tasks in serial
          \State giving each processor at most $1$ virtual-task
        \Else
          \State schedule a level-$0$ task in parallel
          \State distributing its work equally 
        \EndIf
      \EndIf
    \EndWhile
  \end{algorithmic}
\end{algorithm}

We claim the following regarding GGR:
\begin{proposition}
  GGR is $8$-competitive with OPT.
\end{proposition}
\begin{proof}
  hmm
\end{proof}

\section{Lower-Bounds on Competitive Ratio}

In this section we establish that it is impossible for a
deterministic scheduler to get a competitive ratio lower than
$2$. That is, we show that for any deterministic algorithm there
is some input on which OPT has awake time at most half of the
awake time of the deterministic scheduler.

Note that the competitive ratio is trivially at least $1$.

In Table~\ref{tab:lowerboundFork1} and
Table~\ref{tab:lowerboundFork2} we specify two sets of tasks.
For each time we give a list of which tasks arrive in the format
$(\sigma, \pi)\times m$ where $\sigma, \pi$ are the serial and
parallel works of a task and $m$ is how many of this type of task
arrive at this time.

\begin{table}[H]
\caption{}
\label{tab:lowerboundFork1}
\centering
\begin{tabular}{|l|l|}
\hline
time & tasks                    \\ \hline
$0$  & $(4, 2p) \times 1$       \\ \hline
$1$  & $(3, 3p/2) \times (p-1)$ \\ \hline
\end{tabular}
\end{table}

\begin{table}[H]
\caption{}
\label{tab:lowerboundFork2}
\centering
\begin{tabular}{|l|l|}
\hline
time & tasks                    \\ \hline
$0$  & $(4, 2p) \times 1$       \\ \hline
\end{tabular}
\end{table}

Consider an arbitrary deterministic scheduling algorithm. If at
time $0$ the arriving tasks are $(4, 2p)\times 1$ (i.e. a single
task arrives, with serial work $4$ and parallel work $2p$) then
the scheduler has two options: it can schedule this task in
serial, or in parallel.

If no further tasks arrive, i.e. the task schedule is from
Table~\ref{tab:lowerboundFork2} then OPT would have awake time
$2$ by distributing the tasks work equally amongst all
processors, whereas a scheduler that ran the task in serial would
have awake time $4$. In this case the competitive ratio of the
algorithm is at least $2$.

On the other hand, the algorithm could decide to run the task in
parallel. If the algorithm decides to run the task in parallel,
and it turns out that the task schedule is from
Table~\ref{tab:lowerboundFork1}, then the algorithm has again
acted sub-optimally. In particular, for the schedule given in
Table~\ref{tab:lowerboundFork1}, OPT schedules the task that
arrives at time $0$ in serial, and then schedules all the tasks
that arrive at time $1$ in serial as well, and hence achieves
awake time $4$. On the other hand, the awake time of an algorithm
that did not schedule the task that arrived at time $0$ in
serial is at least $5$: such a scheduler may either choose at
time $1$ to cancel the task from time $0$ and run it in serial,
or the scheduler may choose to let the parallel implementation
finish running. In this case the competitive ratio of the
algorithm is $5/4$.

Hence it is impossible for any deterministic algorithm in the
general case of the Serial-Parallel Scheduling Problem, or in
fact in the symmetric-cost-ratio case of the problem, to achieve
a competitive ratio of lower than $1.25$.

By optimizing this argument a bit we can get a stronger
lower-bound of $1.44$ on the competitive ratio (more
specifically, we can get a lower bound of the positive root of
the quadratic $x - 1/x = 3/4$ which is $(3+\sqrt{73})/8 \in
(1.44, 1.45)$).

{\color{red} 
TODO: do a completely different argument to get a better bound?

Are there any good bounds in the symmetric-task case?
}

\section{Randomized Scheduling Algorithms}
Given a particular deterministic scheduling algorithm there will
be some inputs on which the algorithm will perform poorly. 
By employing randomization these worst case inputs can be
mitigated somewhat, at least in expectation.

We propose Algorithm~\ref{alg:rgr}, which we call
\defn{RGR} (RGR stands for \enquote{randomized greedy}), for this case.

\begin{algorithm}
  \caption{RGR}
  \label{alg:rgr}
  \begin{algorithmic}
    \While{True}
      \If{verge time}
        \State sleep for a random amount of time, chosen
        uniformly at random from something, not really sure what 
        \State $q \gets $ number of ready tasks
        \If{$q \ge p/k$}
          \State schedule $\min(q, p)$ tasks in serial
          \State giving each processor at most $1$ task
        \Else
          \State schedule one task in parallel
          \State distributing its work equally 
        \EndIf
      \EndIf
    \EndWhile
  \end{algorithmic}
\end{algorithm}

\begin{proposition}
  The expectation of RGR's competitive ratio on any input
  is at least $1.5$.
\end{proposition}
\begin{proof}
 hmmm. 
\end{proof}


\section{Recursion}
First we must formalize this problem. Like what does
this even mean?

\section{Conclusions}
Greed is good. 
An interesting question is: the whole not doing preemption thing seems pretty dumb? 

\bibliographystyle{plain}
\bibliography{paper}

\end{document}

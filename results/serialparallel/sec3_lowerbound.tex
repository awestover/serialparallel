\section{Lower-Bounds on Competitive Ratio}

In this section we establish that it is impossible for a
deterministic scheduler to get a competitive ratio lower than
$1.44$. That is, we show that for any deterministic algorithm there
is some input on which OPT has awake time at most half of the
awake time of the deterministic scheduler.

Note that the competitive ratio is trivially at least $1$.

In Table~\ref{tab:lowerboundFork1} and
Table~\ref{tab:lowerboundFork2} we specify two sets of tasks.
For each time we give a list of which tasks arrive in the format
$(\sigma, \pi)\times m$ where $\sigma, \pi$ are the serial and
parallel works of a task and $m$ is how many of this type of task
arrive at this time.

\begin{table}[H]
\caption{}
\label{tab:lowerboundFork1}
\centering
\begin{tabular}{|l|l|}
\hline
time & tasks                    \\ \hline
$0$  & $(4, 2p) \times 1$       \\ \hline
$1$  & $(3, 3p/2) \times (p-1)$ \\ \hline
\end{tabular}
\end{table}

\begin{table}[H]
\caption{}
\label{tab:lowerboundFork2}
\centering
\begin{tabular}{|l|l|}
\hline
time & tasks                    \\ \hline
$0$  & $(4, 2p) \times 1$       \\ \hline
\end{tabular}
\end{table}

Consider an arbitrary deterministic scheduling algorithm. If at
time $0$ the arriving tasks are $(4, 2p)\times 1$ (i.e. a single
task arrives, with serial work $4$ and parallel work $2p$) then
the scheduler has two options: it can schedule this task in
serial, or in parallel.

If no further tasks arrive, i.e. the task schedule is from
Table~\ref{tab:lowerboundFork2} then OPT would have awake time
$2$ by distributing the tasks work equally amongst all
processors, whereas a scheduler that ran the task in serial would
have awake time $4$. In this case the competitive ratio of the
algorithm is at least $2$.

On the other hand, the algorithm could decide to run the task in
parallel. If the algorithm decides to run the task in parallel,
and it turns out that the task schedule is from
Table~\ref{tab:lowerboundFork1}, then the algorithm has again
acted sub-optimally. In particular, for the schedule given in
Table~\ref{tab:lowerboundFork1}, OPT schedules the task that
arrives at time $0$ in serial, and then schedules all the tasks
that arrive at time $1$ in serial as well, and hence achieves
awake time $4$. On the other hand, the awake time of an algorithm
that did not schedule the task that arrived at time $0$ in
serial is at least $5$: such a scheduler may either choose at
time $1$ to cancel the task from time $0$ and run it in serial,
or the scheduler may choose to let the parallel implementation
finish running. In this case the competitive ratio of the
algorithm is $5/4$.

Hence it is impossible for any deterministic algorithm in the
general case of the Serial-Parallel Scheduling Problem, or in
fact in the symmetric-cost-ratio case of the problem, to achieve
a competitive ratio of lower than $1.25$.

By optimizing this argument a bit we can get a stronger
lower-bound of $1.44$ on the competitive ratio (more
specifically, we can get a lower bound of the positive root of
the quadratic $x - 1/x = 3/4$ which is $(3+\sqrt{73})/8 \in
(1.44, 1.45)$).

\todo{hypothesis: can get better bounds}
